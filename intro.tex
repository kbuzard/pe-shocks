Much of the work on the political economy of trade agreements focuses on questions of the optimal design of trade agreements, trade agreement negotiations, and trade dispute settlement that arise in the presence of asymmetric information about shocks to an exogenous political economy parameter. The assumption that political economy forces are entirely driven by exogenous forces is a rather drastic simplifying assumption. Do the predictions of our models hold up if this simplifying assumption is lifted?

One of the basic ideas that emerges from this literature is that in the presence of asymmetric information about the strength of the ex-post political economy shocks, it is often advantageous to grant governments a period of relief from trade commitments. That is, one would rather allow a short period of ``escape'' from the agreement rather than have the agreement abandoned forever because domestic political opposition is temporarily too strong to be resisted.

This is an intuitively appealing story, but it turns out that the logic can break down in the presence of endogenous political pressure. An escape clause allows a government to apply a higher tariff barrier when it experiences intense political pressure. But if a government gets a free pass at the WTO whenever it feels sufficient political pressure from domestic interest groups, those interest groups have a strong incentive to exert the required level of pressure regardless of the underlying state of the world, eviscerating the escape clause.

This is one example of a design question whose answers are sensitive to assumptions about the endogeneity of political pressure. In order to examine it and others, I build a model that is comparable to \Textcite{bs2005}, adding an endogenously-determined element to their exogenously-determined political economy weights in a repeated-game setting.

I first establish the baseline case with tariff caps and no escape clause. Here, I find that the presence of endogenous political pressure changes the optimal choice of tariff bindings within a trade agreement in quite significant ways compared with the case where political pressure is assumed to be exogenous. As in \Textcite{mrc2007}, I show that one of the uses of tariff caps is to incentivize the lobby to engage in the political process after the trade agreement is in place. This, perhaps, provides an explanation for the ubiquity of tariff caps. Combined with the idea that governments can use tariff caps to restrain endogenous political pressure, a story emerges in which governments employ trade agreements to carefully manipulate lobbying incentives in order to maximize their political objectives.\footnote{Both points extend results from \Textcite{mrc2007} to the case of perfectly immobile specific factors considered in this paper. To truly extend those results, it must be shown that the results hold when lobbying is possible ex ante---that is, when special interests are able to influence the formation of the trade agreement. The results with ex-ante lobbying are available from the author upon request.}

In order to make this point clear, I compare welfare under the standard \Textcite{baldwin}-style government objective function and a similar, fully-weighted version of the government's objective function. At issue is the fact that the standard government objective function is everywhere increasing in political pressure. In an environment with endogenous political pressure, this leads inexorably to the conclusion that governments never want to discourage special interest groups from exerting pressure. The slight modification to the government objective function demonstrates that governments may indeed want to use trade agreements to reduce lobbying.\footnote{In the environment with only one lobby in each country, the governments will be unable to encourage lobbying in excess of that which is optimal from the lobby's point of view.} Examining this alternative welfare function in combination with endogenous lobbying can provide a bridge between the theoretical literature and the claims of trade policy practitioners that an important role of trade agreements is to rein in protectionist pressure.

In the main example of this paper, I show that the same combination---endogenously-determined political pressure and a general approach to the government's objective function---can explain why the conditions for invoking the WTO Safeguards measure rely on purely observable economic variables, and why the level of protection governments can choose when invoking a Safeguard is related to the injury imposed by the shock and not its political impact. The model formalizes the intuition that if the WTO's Dispute Settlement Body actually followed the procedures suggested by the literature---that is, it enforced commitments based on signals of the political pressure experienced by governments---pressure groups would simply exert more pressure and the Safeguard clause could not play the escape-providing role for which it was intended.

Through the lens of these examples, one can see that modeling choices concerning both the source of political pressure and the form of governments' political objective function have important impacts on questions of the design of and motives behind trade agreements as well as the institutions that help to enforce them. It is important to consider how these features interact with the question being asked. For some inquiries, such as those that are the focus of \Textcite{ms2011, ms2012a} in which the impact of the political economy parameter on the shape of the objective function is not crucial, there may be no loss in employing the simple form. However, for other questions that directly involve the government maximizing with respect to political pressure, a more general treatment may be prudent.\footnote{Note that, although \Textcite{gh94} provide microfoundations for one particular \Textcite{baldwin}-style objective function, it is a very specific form with fixed weights in which $1+a$ is attached to the surplus of groups who lobby, and $a$ to the surplus of those who don't lobby. This cannot microfound a flexible model as in \Textcite{longvousden} in which weights depend on lobbying activity or in which weights are the result of shocks.}

A rich literature has been developed to address questions concerning the design and enforcement of trade agreements. Repeated non-cooperative game models of trade agreements have been considered by \Textcite{mcm86,mcm89}, \Textcite{dixit1987}, \Textcite{bs1990, bs1997a, bs1997b, bs2002}, \Textcite{kovthurs}, \Textcite{maggi99}, \Textcite{ederington}, \Textcite{rosendorff}, \Textcite{bagwell2009}, and \Textcite{park}.

Tariff caps have received particular attention in the literature that takes political pressure to be exogenous. The basic result on the political efficiency of tariff caps established in \Textcite{bs2005} has been extended in numerous directions. \Textcite{hms} show that the basic logic concerning tariff caps is unaltered in a model with contracting costs and multiple policy instruments, while \Textcite{ls} show that the result holds when fines are allowed as punishments, although the politically efficient tariffs may not be achievable when enforcement of the punishment is taken into consideration. 

\Textcite{ab2012} consider that the weight that tariff revenue receives in the government welfare function may be private information and show that tariff caps remain optimal under certain conditions, while \Textcite{ab2013} show that the result goes through when the \Textcite{bs2005} model is generalized to include monopolistic competition as well as more general payoff and distribution function. \Textcite{bb} extend the theory to an environment with asymmetric country sizes with costly state verification and show that tariff caps and escape clauses can both emerge endogenously in optimal trade agreements. \Textcite{bbr} and \Textcite{nos} provide evidence on the relationship between tariff caps, binding overhang and market power that confirms the predictions of the terms of trade theory.

Maggi and Staiger have a series of papers that employ an exogenous political economy force to study questions about the design of trade agreements and trade dispute settlement. \Textcite{ms2012a} study the conditions under which property versus liability rules will be optimal when renegotiation of agreements is possible. They find that when property rules are optimal, agreements are never renegotiated; only liability rules are renegotiated in equilibrium, and when this renegotiation occurs, it always results in trade liberalization. \Textcite{ms2013} builds on this work to answer questions about when governments will settle disputes and how this relates to the contracting environment. \Textcite{ms2011} has a more sophisticated set-up for the exogenous shock that allows the authors to speak to issues of the role of the dispute settlement body as interpreter and completer of incomplete contracts.
		
\Textcite{beshkar2010a} shows that when one assumes that utility is not transferable between countries as has become common in the literature, the optimal mechanism involves less-than-proportional retaliation against parties who have defected from the agreement. \Textcite{beshkar2010b} compares the GATT escape clause to the WTO Safeguards agreement and shows that the DSB as a non-binding arbitrator can assist governments in self-enforcing their trade agreements. \Textcite{martinvergote} demonstrate that future punishment provides for higher welfare than contemporaneous punishment when governments are sufficiently patient. Indeed, they show that retaliation is a necessary feature of any efficient equilibrium in this environment. \Textcite{hungerford} and \Textcite{riezman1991} also consider the impact of different assumptions about reactions and timing of punishments for deviations from agreements.

This work is also related to the literature on the endogenous political economy of trade. The foundational work is \Textcite{gh94}; the insights are applied to trade agreements in \Textcite{gh95}. \Textcite{mrc2007} advanced the literature by demonstrating that there is a domestic commitment role for trade agreements. \Textcite{buzard2013a} features a repeated-game model similar in spirit to the model under consideration here but focusing on questions of optimal punishments when the government has a separation-of-powers structure, while \Textcite{coatesludema} demonstrate that, in the presence of opposition to a trade agreement from foreign lobbies, it may be optimal to liberalize trade unilaterally.

In the next section, I present the model and an in-depth discussion of the political welfare function of the government. Section~\ref{sec:rigid} contains an analysis of tariff caps without the flexibility of escape and introduces the repeated-game environment. Section~\ref{sec:escape} explores questions regarding the design of the escape clause. Section~\ref{sec:concl} concludes.