Much of the work on the political economy of trade agreements focuses on questions regarding the optimal design of trade agreements, trade agreement negotiations, and trade dispute settlement that arise in the presence of asymmetric information about shocks to an exogenous political economy parameter. That political economy forces are entirely driven by exogenous forces is a rather drastic simplifying assumption. Do the predictions of our models hold up if this simplifying assumption is relaxed?

One of the basic ideas that emerges from this literature is that in the presence of asymmetric information about the strength of the ex-post political economy shocks, it is often advantageous to grant governments a period of relief from trade commitments. That is, one would rather allow a short period of ``escape'' from the agreement rather than have the agreement violated because domestic political opposition is temporarily too strong to be resisted.

This is an intuitively appealing story, but the logic can break down in the presence of endogenous political pressure. An escape clause allows a government to apply a higher tariff barrier when it experiences intense political pressure. But if a government gets a free pass at the WTO whenever it feels sufficient political pressure from domestic interest groups, those interest groups have a strong incentive to exert the required level of pressure regardless of the underlying state of the world, eviscerating the escape clause.

This is one example of a design question whose answers are sensitive to assumptions about the endogeneity of political pressure. In order to examine this question and others, I use a simple model that is standard in the literature (e.g. \Textcite{bs2001}, \Textcite{bown2002}, \Textcite{bown2004}, \Textcite{bs2005}, \Textcite{martinvergote}, \Textcite{bagwell2009}, \Textcite{beshkar2010a}, \Textcite{ms2012a}, and \Textcite{ms2013}) and add an endogenously-determined element to the usual exogenously-determined political economy weights. Because of its tractability, this model can address a wide range of institutional design questions while also allowing endogenous politics to be taken into account in a repeated-game setting.\footnote{Repeated non-cooperative-game models of trade agreements have been considered by \Textcite{mcm86,mcm89}, \Textcite{dixit1987}, \Textcite{bs1990, bs1997a, bs1997b, bs2002}, \Textcite{kovthurs}, \Textcite{maggi99}, \Textcite{coatesludema}, \Textcite{ederington}, \Textcite{rosendorff}, \Textcite{bagwell2009}, and \Textcite{park}. \Textcite{buzard2013a} features a repeated-game model similar in spirit to the model under consideration here but focusing on questions of optimal punishments when the government is non-unitary.}

I first establish the baseline case with tariff caps and no escape clause. Here, I find that the presence of endogenous political pressure changes the optimal choice of tariff bindings within a trade agreement in quite significant ways compared with the case where political pressure is assumed to be exogenous, and that endogenous political pressure makes it harder to incentivize cooperation through repeated game incentives. As in \Textcite{mrc2007}, I show in this simpler framework that allows for easy comparison to the exogenous shocks case that one of the uses of tariff caps is to incentivize the lobby to engage in the political process after the trade agreement is in place. Combined with the idea that governments can use tariff caps to restrain endogenous political pressure, a story emerges in which governments employ trade agreements to carefully manipulate lobbying incentives in order to maximize their political objectives.

In order to make this point clear, I compare political welfare under the standard \Textcite{baldwin}-style government objective function to political welfare under the government objective function I introduce that takes account of endogenous politics. In my formulation, the government is not necessarily always better off as political pressure climbs higher and higher, while the \Textcite{baldwin}-style  government objective function is everywhere increasing in political pressure. \Textcite{gh94} and its extension to trade agreements in \Textcite{gh95} provide microfoundations for this \Textcite{baldwin}-style government objective function, where a fixed weight of $1+a$ is attached to the surplus of groups who lobby, and $a$ to the surplus of those who don't lobby.\footnote{Note that $a$ is the weight the government places on social welfare.} This cannot microfound a flexible model as in \Textcite{longvousden} or in which shocks to the political-economy weights come from shocks to any parameter other than the weight the government places on social welfare.

More importantly, this standard functional form leads inexorably to the conclusion that governments do not want to discourage political pressure except in the presence of long-run inefficiencies such as in \Textcite{mitra} and \Textcite{mrc2007}. The modification to the government objective function I propose leads to the result that governments may indeed want to use trade agreements to reduce lobbying in many environments.\footnote{With only one lobby in each country, the governments will be unable to encourage lobbying in excess of that which is optimal from the lobby's point of view.} Examining this alternative welfare function in combination with endogenous lobbying can provide a bridge between the theoretical literature and the claims of trade policy practitioners that an important role of trade agreements is to rein in protectionist pressure.

In the main example of this paper, I show that the same combination---endogenously-determined political pressure and a general approach to the government's objective function---can explain why the conditions for invoking the WTO Safeguards measure rely on purely observable economic variables, and why the level of protection governments can choose when invoking a Safeguard is related to the injury imposed by the shock and not its political impact. The model formalizes the intuition that if the WTO's Dispute Settlement Body actually followed the procedures suggested by the literature---that is, it enforced commitments based on signals of the political pressure experienced by governments---pressure groups would simply exert more pressure and the Safeguard clause could not play the escape-providing role for which it was intended. On the other hand, if the purpose of contingent protection measures is actually to provide a political safety valve, this analysis could help explain the infrequency with which safeguards are invoked under the WTO, as I argue that the WTO's escape clause has been designed to provide relief only from economic shocks.

We miss important dynamics like this one when we assume political pressure is exogenous as is done for tractability in almost all the literature that studies trading rules and institutions (see \Textcite{mrc2007}, \Textcite{lt} and \Textcite{buzard2013a} for exceptions). The main contribution of this paper is to provide a tractable framework for integrating lobbying incentives into these analyses. This framework can be used to study a wide range of questions, including dispute settlement design, optimal retaliation schemes and depth of integration among others. It's well-positioned to examine the interactions between exogenous shocks and lobbying incentives and will be useful for shedding light on questions surrounding how governments choose among protective measures over time.

Through the lens of the examples considered in this paper, one can see that modeling choices concerning both the source of political pressure and the form of governments' political objective functions have important impacts on questions of the design of and motives behind trade agreements as well as the institutions that help to enforce them. It is important to consider how these features interact with the question being asked. For some inquiries, such as those that are the focus of \Textcite{ms2011, ms2012a} in which the impact of the political economy parameter on the government's objective function is not crucial, there may be no loss in employing a simple form with exogenous political pressure. However, for other questions in which lobbying incentives might be impacted or the government's desire for a commitment mechanism is central, a more general treatment seems prudent.

%A rich literature has been developed to address questions concerning the design and enforcement of trade agreements.

\textbf{A few additional comments on the literature}. Tariff caps have received particular attention in the literature that takes political pressure to be exogenous. The basic result on the political efficiency of tariff caps established in \Textcite{bs2005} has been extended in numerous directions. \Textcite{hms} show that the basic logic concerning tariff caps is unaltered in a model with contracting costs and multiple policy instruments, while \Textcite{ls} show that the result holds when fines are allowed as punishments, although the politically efficient tariffs may not be achievable when enforcement of the punishment is taken into consideration. 

\Textcite{ab2012} consider that the weight that tariff revenue receives in the government welfare function may be private information and show that tariff caps remain optimal under certain conditions, while \Textcite{ab2013} show that the result goes through when the \Textcite{bs2005} model is generalized to include monopolistic competition as well as more general payoff and distribution function. \Textcite{bb} extend the theory to an environment with asymmetric country sizes with costly state verification and show that tariff caps and escape clauses can both emerge endogenously in optimal trade agreements. \Textcite{bbr} and \Textcite{nos} provide evidence on the relationship between tariff caps, binding overhang and market power that confirms the predictions of the terms of trade theory.

%Maggi and Staiger have a series of papers that employ an exogenous political economy force to study questions about the design of trade agreements and trade dispute settlement. \Textcite{ms2012a} study the conditions under which property versus liability rules will be optimal when renegotiation of agreements is possible. They find that when property rules are optimal, agreements are never renegotiated; only liability rules are renegotiated in equilibrium, and when this renegotiation occurs, it always results in trade liberalization. \Textcite{ms2013} builds on this work to answer questions about when governments will settle disputes and how this relates to the contracting environment. \Textcite{ms2011} has a more sophisticated set-up for the exogenous shock that allows the authors to speak to issues of the role of the dispute settlement body as interpreter and completer of incomplete contracts.
		
On escape clauses, \Textcite{bs1990} is similar in spirit to \Textcite{bs2005}. Here, there is no private information and trade volume shocks in the presence of self-enforcement constraints are behind the need for an escape clause. \Textcite{bc} find empirical support for this theory. \Textcite{beshkar2010b} compares the GATT escape clause to the WTO Safeguards agreement and shows that the Dispute Settlement Body as a non-binding arbitrator can assist governments in self-enforcing their trade agreements.

In the next section, I present the model and an in-depth discussion of the political welfare function of the government. Section~\ref{sec:rigid} contains an analysis of tariff caps without the flexibility of escape and introduces the repeated-game environment. Section~\ref{sec:escape} explores questions regarding the design of the escape clause. Section~\ref{sec:concl} concludes.