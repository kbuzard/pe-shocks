\documentclass[12pt]{article}

\addtolength{\textwidth}{1.4in}
\addtolength{\oddsidemargin}{-.7in} %left margin
\addtolength{\evensidemargin}{-.7in}
\setlength{\textheight}{8.5in}
\setlength{\topmargin}{0.0in}
\setlength{\headsep}{0.0in}
\setlength{\headheight}{0.0in}
\setlength{\footskip}{.5in}
\renewcommand{\baselinestretch}{1.0}
\setlength{\parindent}{0pt}
\linespread{1.1}

\usepackage{amssymb, amsmath, amsthm, bm}
\usepackage[normalem]{ulem}
\usepackage{graphicx,csquotes,verbatim}
\usepackage[backend=biber,block=space,style=authoryear]{biblatex}
\setlength{\bibitemsep}{\baselineskip}
\usepackage[american]{babel}
%dell laptop
\addbibresource{C:/Users/Kristy/Dropbox/Research/xBibs/tradeagreements.bib}
%\addbibresource{C:/Users/Kristy/Documents/Dropbox/Research/xBibs/tradeagreements.bib}
\renewcommand{\newunitpunct}{,}
\renewbibmacro{in:}{}

\usepackage[pdftex,
bookmarks=true,
bookmarksnumbered=false,
pdfview=fitH,
bookmarksopen=true,hyperfootnotes=false]{hyperref}


\DeclareMathOperator*{\argmax}{arg\,max}
\usepackage{xcolor}
\hbadness=10000

\newcommand{\ve}{\varepsilon}
\newcommand{\ov}{\overline}
\newcommand{\un}{\underline}
\newcommand{\ta}{\theta}
\newcommand{\al}{\alpha}
\newcommand{\expect}{\mathbb{E}}
\newcommand{\ga}{\gamma}
\newcommand{\Ga}{\Gamma}
\newcommand{\de}{\delta}

\begin{document}
\section{Feedback on Apr. 4, 2016 draft}

\subsection{Rochester Seminar, April 6, 2016 (mostly Mark Bils)}
\begin{itemize}
	\item Need to say that I'm ruling out side payments so it's clear that there can't just be a menu of punishments that exactly keeps each government in line
		\begin{itemize}
			\item Risk (average) vs. incentives (marginal) impossibility result --- but with no money transfers, even this breaks down
			\item Can I think of this as an insurance problem between the governments?
		\end{itemize}
	\item Show what the optimum is, then why they can't get it, how they have to try to mimic it with these instruments (tariff caps, escape clause, these types of punishments, etc.)
	\item Going from $\ga(s)$ to $\ga(e)$ --- essentially taking away the structure of the $s$'s was confusing b/c the policy space doesn't make any sense
\end{itemize}

\subsection{From SITE 2016 presentation}
\begin{itemize}
	\item Kyle: McLaren in his chapter calls the Grossman and Helpman setup with contributions 'felonious'
	\item Kyle: why doesn't government (dis)value expenditure by lobby? Needs to be counted in government welfare function or justify why not
		\begin{itemize}
			\item Rohan: $(1-a)\cdot C + a \cdot W \equiv C + a \cdot (W-C)$
			\item When weights on $\pi_X$ are not linear, what are the implications for resource waste? Is it the same as $+ C / - C$?
		\end{itemize}
	\item $\ga(e)$
		\begin{itemize}
			\item Giovanni: think of expost $\ga(e)$ as effort to elect/choose ex-post decision-maker
			\item At worst, I can use leg / exec split to deal with this criticism
			\item Kyle (???): advertising paper (???)
			\item Mostafa, Nuno, Giovanni, Anson to some extent: Cardinal vs. ordinal
				\begin{itemize}
					\item Giovanni: if I can show properties survive  monotonic transformation, I'm fine
					\item They don't like gov't choosing its own preferences. May just need to change wording
				\end{itemize}
			\item Bob: what is fundamental structure of preferences?
			\item The way $e$ affects preferences doesn't change!!! Preferences are wrapped up in $\ga$!
		\end{itemize}
	\item Bob: maybe gov't has some other reason for setting non-zero tariff that opens the door for lobbies
		\begin{itemize}
			\item Maybe I should just get rid of tariff cap result altogether
		\end{itemize}
	\item David DeRemer: Safeguards have dropped off in use around the world
		\begin{itemize}
			\item Look at U.S. vs. ROW (he sent papers)
		\end{itemize}
	\item I think that I need to bring people along more carefully from GH/MRC to what I'm doing
	\item Bob and Kyle: take my time, make it as good as possible. Better to have quality.
\end{itemize}


\newpage
\section{Feedback on Jan. 10, 2015 draft}

\subsection{Chad Bown, Jan. 22, 2015}
\begin{itemize}
	\item What it's missing: intersection with reality
		\begin{itemize}
			\item Anecdotes
			\item How policy is used
				\begin{itemize}
					\item I thought that escape clause implications \textit{were} this
				\end{itemize}
			\item Something like how GH predicts export taxes but it doesn't work that way in the real world
			\item Well articulated question / paradox
		\end{itemize}
	\item Look at ``Theory of Managed Trade'' model motivation (BS)
	\item Dangerous to build on JLS paper; better to connect to bigger cites or will limit impact, where it can be published
		\begin{itemize}
			\item Perhaps higher level model like BS GE model or some alternative approach that is not so rinky dink (my words)
			\item You can have one good idea in a paper, the rest get lost
		\end{itemize}
	\item How did Devashish structure his AER paper: the argument, the model?
		\begin{itemize}
			\item This paper is structured differently than the norm---is there a good reason to do it?
			\item Don't want to challenge what people are used to unless I really need to
		\end{itemize}
	\item In general, how do AER papers \textit{look}?
	\item Intro doesn't reflect body of paper
	\item What kinds of empirical questions might we ask with this model?
		\begin{itemize}
			\item What would be different because we use this model?
		\end{itemize}
\end{itemize}

\newpage
\subsection{Ben Zissimos, May 11, 2015}

\begin{itemize}
	\item I think the topic of tariff caps and how they can be motivated by endogenising lobbying effort is a very interesting one and this literature seems to be wide open.
	\item I also think it is a nice approach to extend the Bagwell-Staiger-2005 framework by endogenising their gamma parameter.
	\item You do a lot in the paper, but for me the really interesting contribution is the one you highlight in the last couple of sentences of the abstract:
\begin{quote}
The presence of endogenous politics can also destroy an escape clause's ability to provide flexibility in times of large negative political shocks when lobbies use the flexibility to seek rents. This can explain why use of WTO Safeguards are conditioned on measurable economic indicators as well as why Safeguard levels of protection are not regulated.
\end{quote}
\end{itemize}
 
 
Here are some thoughts on which particular aspects of your paper I got excited about and why.
\begin{enumerate}
	\item I started to get really excited about the paper around page 19, especially with Result 3.  The part I really liked was when you analysed the exogenous and endogenous effects on gamma simultaneously around page 28.  In fact I was itching to get to that part right from when you discussed it in the introduction.  I also found the results in the static framework reasonably predictable.  So I'm wondering whether you could dispense with most of the static analysis and get into the repeated game more quickly.
 
	\item What I found most interesting was the effect you discussed on the 2nd paragraph of page 28. 
 
``There is a large parameter space over which an escape clause could not function in the way it is intended. When the realization of the shock is high, the government will report the high shock and apply the escape clause tariff. But when the shock is low, the lobby will exert effort so that the sum of gamma(s) and gamma(e) equals the high realization. From the point of view of maximizing the political welfare of the governments, this is only problematic if, as demonstrated in Section 5, the governments' ex-ante welfare is reduced by excess lobbying.''
 
I think the effect the paper identifies here is critical to the operation of trade policy making under the rules of the WTO and to the best of my knowledge it has not received attention in the literature.  I would like to see the paper do much more with this.  Here are some thoughts on how you could do this.
 
First, the paper makes the extremely important point above almost in passing.  I would be interested to see a more detailed examination of how this works.  It's particularly interesting that in the presence of a shock the governments will use the escape clause as intended but without a shock a lobby group would always `step into the breach' and cause it to be abused.  I would like to see the paper devote more attention to the conditions under which this problem arises and when it does not.
 
Then the paper begins to address the equally important question of what can be done about this problem, but again stops short of providing a really compelling answer.  I think it would be worth exploring this in detail.  The paper does get to the heart of the issue, which is that `s' can be observed by the DSB but e cannot.  But the paper then moves on to ex-ante lobbying before getting to the bottom of how to design a dispute settlement system that resolves the conflict created by `e vs s'.
 
	\item Following on from the previous point, the one institutional feature I kept expecting the paper to talk about was a `withdrawal of equivalent concessions' (WEC).  As you probably know, the only legal response to a breach of an agreement under the GATT/WTO is WEC: if your partner denies you $\$$1m worth of market access promised to you under an agreement then that's what you're allowed to withdraw from them.  One interesting feature of WEC highlighted by Zissimos (2007) is that a government `chooses the severity of its own punishment' by the extent of its initial deviation.  In that paper, I took WEC for granted and then played out trade liberalization according to its rules.  But this left open the question of why WEC made sense as an approach to punishment.  Now I'm wondering whether your framework offers an answer.  Under WEC, it is worth deviating from the agreement in proportion to `s' because your partner deviates to the same extent and that keeps the agreement on track; the point first made by Bagwell and Staiger (1990).  But WEC might eliminate the incentive to respond to lobbying pressure `e' because when your trade partner deviates by WEC this takes away from you exactly (in a symmetrical framework) what you gained from the lobby for implementing a deviation of that size.
 
\textbf{I would find an examination of WEC much more compelling that the approach to punishment that you currently discuss on page 20, whereby two bindings are negotiated: one that applies if the trade partner keeps to the agreement and one that applies if they deviate.  As far as I know this latter approach does not mirror at all the institutional features actually set out by the GATT/WTO.}
 
	\item I like the way you use Bagwell and Staiger (2005) as your benchmark.  But I also wondered whether Maggi and Rodriguez-Clare (2007, henceforth MRC) also offered a useful benchmark.  I know you did have comparisons to their paper in many places but it would be useful if this could be more systematic, and if your paper could draw careful parallels to your assumptions and theirs.  To keep things manageable, I wonder whether you could restrict attention to the case of ex post lobbying.  This is where all the action is, right?  In both papers, the government wants to set a tariff cap in order to be able to extract rents from lobbies afterwards.  But your mechanisms seem quite different.  They have capital mobility whereas your factors are specific to each sector.  If capital is highly immobile they have a hold-up problem, which creates an incentive for lobbies to try to prevent too much liberalization under the agreement.  What drives this feature in your model?  Here I think it would be useful to explain carefully the sequence of events in your set-up more carefully.  What is the relationship between lobby effort level and tariff policy?  Is effort fixed first, after which the government sets its tariff?  If so it seems to me that would drive a kind of hold-up because lobbying effort is sunk.  It would be useful to be clearer about this so we can really understand where the results are coming from and how they compare to MRC.  Having done that, you could bring in ex-ante lobbying and treat it the same way.      
 
	\item I'm wondering how you view the time-frame over which your model is set.  On one hand a specific-factors set-up is normally associated with the short-run.  But on the other hand you have an infinite time horizon which suggests very long run.
 
	\item Have you seen the work that Beshkar and Bond have done on what they call `contingent protection'?  I can't remember if they are drawing on the work of other people when they refer to the escape clause as `costly state verification.'  Their idea is that if you want to use the escape clause then this comes at the cost of providing information to the GATT panel to persuade them that you really have been hit by a shock.  This seems to provide an alternative way of separating genuine shocks from those created by lobby groups.  How does your approach relate to costly state verification?
 
	\item I'm wondering whether footnote 18 provides a general way of addressing the issues arising from the endogeneity of gamma that you illustrate in Figures 1$\&$2.
\end{enumerate}

\vskip.3in
Minor comments
\begin{enumerate}
	\item Page 1:  Should define gamma before using it, especially given how critical it is to the paper.
	\item \sout{Page 17: `is invariant'} DONE
	\item \sout{Page 26: `an interior'} DONE
	\item \sout{I'm wondering why are the results referred to as `Results' and not `Propositions'?  Odd question when I put it like that!  But since we're used to `Propositions', saying `Results' is a bit distracting.  Or maybe I'm just being old fashioned ;-)} DONE
\end{enumerate}

\newpage
\subsection{Maurizio Zanardi, June 22, 2015}
\begin{itemize}
	\item At the moment, I find the paper a bit unstructured in that it does a lot but it doesn't sell in a neat and clear way what it is doing.
	\item Starting from the introduction, I think you make a convincing case that it is important to endogenize the extent of political pressure and I believe this would be THE selling point of the paper. However, you have not completely sold your contribution and its novelty that you bring in the issue of the objective function of the government. Is this aspect crucial for the point you want to make? If yes, it should be integrate with the previous object from the beginning. If not, it should be postponed a bit.
	\item I also find the introduction too long. You could/should split the literature review that begins on page 4 to a separate section. And in any case, I found that you don't engage with the literature in terms of how you contribute/differentiate from it. Again, your main point seems to get lost into the discussion instead of standing out.
	\item If above I suggest to add a section, I think that the paper is already too long and divided into too many sections but that goes back to my first comment about identifying the key issue/contribution: endogeneity of political pressure and/or objective function. The analysis is also split between strict and weak commitments. Which scenario clearly develops the point/contribution you want to make? Shouldn't the other one occupy a smaller place (i.e. space) of the paper?)
	\item Finally, introducing ex-ante lobbying is not simply an "extension"... it seems to me a more fundamental issue in itself: should it be developed in a separate paper? Should the intuition only developed in the current paper (and the formal analysis moved to an appendix)?
	\item In some parts the paper also seem repeating itself: you review Bagwell and Staiger (2005) in Section 2.2 but you provide another summary of the same paper in Section 3. Again, a revised structure may reduce the overlap, make the paper shorter, and be more to the point (and in general it seems to me that you seem to "dependent" on Bagwell and Staiger... (e.g. beginning of Sections 4 and 6).
	\item In a sort of conclusion, I think that you need to think a bit more about the objective(s) of the paper and how best to achieve them. And I hope that some of the above is useful!
\end{itemize}


\newpage
\section{Feedback on 2014 draft}
\subsection{May 30, 2014, Spring Midwest meetings}
\begin{itemize}
	\item Rick Bond, Ian Wooton, James Lake, Mostafa, David Kuenzel at the talk
	\item James: need to be more precise about when $\ga$ is evaluated
\end{itemize}

\newpage
\section{Presentation Notes}
For SITE 2016
\begin{itemize}
	\item Ethier 2012 for protection decreasing function of $\tau$ [NEED TO REVISIT IN TEXT]. ``The Political-Support Approach to Protection,'' Global Journal of Economics
		\begin{itemize}
			\item Making $W$ quadratic in $\ga$ such as $\ga - \ga^2$ gives a nice interior max
		\end{itemize}
	\item $\ga(s)$ comes from MS 2011 QJE, ``The Role of Dispute Settlement Procedures in Trade Agreements''
	\item BS approach from 2002 book originates from 1999 AER ``An Economic Theory of GATT''
	\item Add title, axis to graph
	\item Review DSB literature (slide 13)
	\item Review my ex-ante results
\end{itemize}

\end{document}