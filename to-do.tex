\documentclass[12pt]{article}

\addtolength{\textwidth}{1.4in}
\addtolength{\oddsidemargin}{-.7in} %left margin
\addtolength{\evensidemargin}{-.7in}
\setlength{\textheight}{8.5in}
\setlength{\topmargin}{0.0in}
\setlength{\headsep}{0.0in}
\setlength{\headheight}{0.0in}
\setlength{\footskip}{.5in}
\renewcommand{\baselinestretch}{1.0}
\setlength{\parindent}{0pt}
\linespread{1.1}

\usepackage{amssymb, amsmath, amsthm, bm}
\usepackage[normalem]{ulem}
\usepackage{graphicx,csquotes,verbatim}
\usepackage[backend=biber,block=space,style=authoryear]{biblatex}
\setlength{\bibitemsep}{\baselineskip}
\usepackage[american]{babel}
%dell laptop
\addbibresource{C:/Users/Kristy/Dropbox/Research/xBibs/tradeagreements.bib}
%\addbibresource{C:/Users/Kristy/Documents/Dropbox/Research/xBibs/tradeagreements.bib}
\renewcommand{\newunitpunct}{,}
\renewbibmacro{in:}{}


\usepackage[pdftex,
bookmarks=true,
bookmarksnumbered=false,
pdfview=fitH,
bookmarksopen=true,hyperfootnotes=false]{hyperref}

\DeclareMathOperator*{\argmax}{arg\,max}
\usepackage{xcolor}
\hbadness=10000

\newcommand{\ve}{\varepsilon}
\newcommand{\ov}{\overline}
\newcommand{\un}{\underline}
\newcommand{\ta}{\theta}
\newcommand{\al}{\alpha}
\newcommand{\expect}{\mathbb{E}}
\newcommand{\ga}{\gamma}
\newcommand{\Ga}{\Gamma}
\newcommand{\de}{\delta}

\begin{document}
\begin{center}
PE Shocks To-Do list
\end{center}

\section{``Dynamic preferences''}
	\begin{itemize}
					\item Gul and Pesendorfer's (2001) Econometrica is a two-period model. No dynamic inconsistency; derive temptation and desire for commitment in a model with dynamically consistent preferences.
					\item Even better, especially advertising example (credit Kyle): Becker, Gary. S. and George J. Stigler, ``De Gustibus Non Est Disputandum,'' The American Economic Review, Vol. 67, No. 2 (Mar., 1977), pp. 76-90
					\item Simpler: Becker, G. S. and K. M. Murphy (1993), Simple Theory of Advertising as a Good or Bad, Quarterly Journal of Economics, 942-64.
						\begin{itemize}
							\item Their firms are my firm. Their consumers are my government
							\item Kyle's lit review talkes about `persuasive,' `informative,' and `complementary' views of advertisting. This is complementary.
							\item $e$ and $\tau$ enter objective function. Gov't is sophisticated, can make commitment that alters $e$ (i.e. behavior of another actor). But $CS + \ga(e)PS + \ldots$, just like $u(X,Y,A)$; says consumer could work to pass law about number of adds on TV each hour.
							\item Use newly drawn extensive form to explain this, lays out when welfare is realized
							\item Addresses James' concern about needing to be more precise about when $\ga$ is evaluated)
			\begin{itemize}
				\item Skeptical of comparative-static style result: can I really compare governments with different $\ga$'s?
			\end{itemize}

		
				\end{itemize}
			\item Mostafa, Nuno, Giovanni, Anson to some extent: Cardinal vs. ordinal
				\begin{itemize}
					\item Giovanni: if I can show properties survive  monotonic transformation, I'm fine
					\item They don't like gov't choosing its own preferences. May just need to change wording
				\end{itemize}
			\item Bob: what is fundamental structure of preferences?
			\item The way $e$ affects preferences doesn't change!!! Preferences are wrapped up in $\ga$!
				\begin{itemize}
					\item ``How the assumption of a non-unitary legislature translates into this incentive constraint. In the non-unitary legislature, if the current period incentive constraint includes any future value of lobbying effort, this would mean the current-period median legislator evaluating her current-period incentive constraint using a mixture of her own political-economy weight with those of the legislators who are median in the future in the trade agreement and trade-war scenarios. The first major step I have taken to clarify the incentive constraint is the analysis in Appendix B.1 of the unitary model (described above), which I would guess is the model you had in mind. I discuss the differences between the unitary and non-unitary models in the beginning of that Appendix.''
					\item ``Note that the median legislator, whose identity is determined by the lobby's effort level $e_b$, evaluates future payoffs according to her own political economy weight, $\ga(e_b)$. Of course, depending on legislator $e_b$'s choice, either legislator $e_a$ or legislator $e_{tw}$ will be the decision maker in those future periods. But legislator $e_b$, who is the decision maker in the current period, maximizes her own welfare given the predicted behavior of future decision makers.''

				\end{itemize}
			\item Devashish: Utility fcn didn't change. Just got evaluated at a different point
				\begin{itemize}
					\item Preferences don't change. It's persuasion that affects decision-making. 2nd story: Giovanni's ``expost $\ga(e)$ as effort to elect/choose ex-post decision-maker''
				\end{itemize}
		\end{itemize}
		

\section{Lobbying cost included in gov't welfare}
	\begin{itemize}
		\item Rick Bond: need lobbying to use up resources: if it changes net profits, it should also change $\pi$ in $W(\tau,\tau^*)$
		\item Kyle: why doesn't government (dis)value expenditure by lobby? Needs to be counted in government welfare function or justify why not
			\begin{itemize}
				\item Rohan: $(1-a)\cdot C + a \cdot W \equiv C + a \cdot (W-C)$
				\item When weights on $\pi_X$ are not linear, what are the implications for resource waste? Is it the same as $+ C / - C$?
			\end{itemize}
		\item Try subtracting $e$ and see what it does to the graphs. Does it give me the kind of non-monotonicity I'd like? YES
			\begin{itemize}
				\item Does it do anything to the analysis?
			\end{itemize}
		\item JIE$\_$revision: I have also verified that \emph{none} of the results of the model change if lobbying effort is subtracted from the legislative/executive welfare function (except, of course, the evaluated level of welfare). I have refrained from adding the ``$-e$'' to the welfare function out of a desire to be consistent with the existing literature.
	\end{itemize}


\section{BS2005 vs. MRC2007}
	\begin{itemize}
		\item Dangerous to build on JLS paper; better to connect to bigger cites or will limit impact, where it can be published
			\begin{itemize}
				\item Perhaps higher level model like BS GE model or some alternative approach that is not so rinky dink (my words)
			\end{itemize}
		\item I like the way you use Bagwell and Staiger (2005) as your benchmark.  But I also wondered whether Maggi and Rodriguez-Clare (2007, henceforth MRC) also offered a useful benchmark.  I know you did have comparisons to their paper in many places but it would be useful if this could be more systematic, and if your paper could draw careful parallels to your assumptions and theirs.  In both papers, the government wants to set a tariff cap in order to be able to extract rents from lobbies afterwards.  But your mechanisms seem quite different.  They have capital mobility whereas your factors are specific to each sector.  If capital is highly immobile they have a hold-up problem, which creates an incentive for lobbies to try to prevent too much liberalization under the agreement.  What drives this feature in your model?  Here I think it would be useful to explain carefully the sequence of events in your set-up more carefully.  What is the relationship between lobby effort level and tariff policy?  Is effort fixed first, after which the government sets its tariff?  If so it seems to me that would drive a kind of hold-up because lobbying effort is sunk.  It would be useful to be clearer about this so we can really understand where the results are coming from and how they compare to MRC.  Having done that, you could bring in ex-ante lobbying and treat it the same way. 
		\item In general it seems to me that you seem to "dependent" on Bagwell and Staiger... (e.g. beginning of Sections 4 and 6).
		\item DGH / MRC analogy from JIE$\_$revision
			\begin{itemize}
				\item Footnote fn:mrc1 is good place to start to put references to MRC 2007
				\item I've supplied the suggested results to connect my model to that of Dixit, Grossman and Helpman (1997) in Appendix B.2, along with some text pointing to it at the end of Section 2.2 on page 10. The results from the DGH-style model are qualitatively different in a minor way. I discuss this difference in Appendix B.1 where I present a unitary government/legislature version of the model that most closely matches Maggi and Rodriguez-Clare 2007.
				\item The model does admit an interpretation in which the same branch of government both negotiates the trade agreement and decides on the applied tariff ex-post. In this sense, the structure of the one-shot game shares much in common with Maggi and Rodriguez-Clare 2007.
				\item The fact that the applied tariffs are equal to the negotiated binding is reminiscent of Maggi and Rodriguez-Clare 2007. Exactly the same dynamic is at play here: specifying trade agreement tariffs as caps instead of strong bindings keeps the lobby active during periods where the trade agreement is honored. In Appendix B.3, I analyze the model with strong bindings and show that the results would be altered in magnitude but not in spirit by assuming that the trade agreement tariffs are strong bindings instead of tariff caps. The only change to the model is that under strong bindings there would be zero lobbying effort during a trade agreement phase as the lobby would not need to exert effort to bid protection levels up to the trade agreement tariff. There would still be no disputes in equilibrium, but the lobbying constraint is easier to satisfy with strong bindings because the gap between trade war and trade agreement profits shrinks when the lobby stops exerting effort to receive the trade agreement tariff. Because lower trade agreement tariffs can be sustained under strong bindings, executive and legislators with small political economy weights prefer a strong-binding agreement and legislators with high political economy weights prefer a weak-binding agreement.
				\item Consider the following, alternative interpretation of the model. Assume there is only one decision-making body but the lobby is not active during the ex-ante phase, that is, when the trade agreement is being negotiated. For ease of exposition, take this single decision-making body to be the legislature so that the single decision-making body has the preferences in Expression 2. At the ex-ante stage, the welfare function would reduce to that in Expression 4. This interpretation fits into the framework of Maggi and Rodriguez-Clare 2007 by assuming that capital is perfectly mobile in the long run so that it is not worthwhile for the lobby to expend resources to influence the negotiation of the trade agreement.\footnote{To match the assumption of a social-wefare maximing executive, this requires the additional assumption that $\ga(0)=1.$ The model is qualitatively unchanged for other values of $\ga(0)$ as long as the analogue of Assumption 2 below holds.} Note that this remains a non-unitary model. Out of a single decision-making body, the level of lobbying effort determines a different decisive member depending on the situation, e.g. during ex-ante negotiations ($e=0$) versus a trade war ($e=e_{tw}$). A unitary model---in which a single actor makes different decisions depending on how much lobbying effort she experiences---results in minor, qualitative changes to the results. I discuss the unitary version of the model in Appendix B.1. Note that in either case, it is only the realization of $\ga(\cdot)$ that changes with $e$, not the preferences themselves which are embodied in $\ga(\cdot)$.
				\item The model admits this `one-branch' interpretation only if the one branch is non-unitary
				\item the model analyzed in the body of this paper can be given an interpretation in line with Maggi and Rodriguez-Clare 2007 if capital is completely mobile in the long run and the single decision-making body is non-unitary. In Appendix B.1, I compare the results of this paper to those from a model with a unitary government, as this would seem to bring the model most closely into alignment with Maggi and Rodriguez-Clare 2007 and a large part of the literature. Then, in Appendix B.2, since the government welfare function used in this paper can be interpreted as a special case of the one proposed by DGH97, I will show that a model patterned after DGH97 and specialized to this environment---which is also a unitary model---provides results that are qualitatively similar to the unitary model.
				\item I have added Appendix B.1 with analysis of the unitary model
					\begin{itemize}
						\item In the model with a non-unitary government, the lobby's effort $e$ determines the identity of the decision maker. Thus the decision maker at the time the lobby is pushing for the agreement to be broken ($e=e_b>e_a$ in order to have a median legislator who will break the agreement) is different from the decision maker during the trade war phase $(e = e_{tw})$ and the decision maker during a trade agreement phase ($e=e_a(\tau^a)$). The weight on the lobbying industry's profits changes with lobbying effort because lobbying effort determines a different median voter in the legislature or government more generally. However, when the median voter during a `break' phase is evaluating her incentive constraint, she values future tariff choices --- which will be determined by whoever is median in the future --- with her own preferences. Although she will not be the decision maker in the future, there is no reason for her to evaluate future welfare according to some other legislator's preference.
						\item If the model is interpreted as having a unitary government, there is only one decision maker. There is a fixed mapping from lobbying effort to the weight the decision maker places on the lobby's profits, and the realization of this weight changes with the realization of lobbying effort. Thus, predicting future lobbying effort, the decision maker knows what the realization of her weight on the lobby's profits will be and evaluates the incentive constraint accordingly. Expression (10) would be modified while the lobbying constraint is unchanged.
					\end{itemize}
				\item Appendix B.2 with analysis of DGH model
					\begin{itemize}
						\item I claim that the government welfare function presented in this paper can be interpreted as a special case of the general welfare function proposed in DGH97. They specify government welfare as $G(a,c)$ where $a$ is the policy vector $\left(\left(\tau,\tau^*\right)\right)$ and $c$ is a vector of payments from each lobby ($e$).
						\item Here I examine a version of the DGH97 model specialized to this environment. Because the DGH97 model is fundamentally a unitary government model, the relevant benchmark is the unitary version of the model. The results of the two unitary models are qualitatively the same.
						\item To be clear about the differences, the non-unitary model examined in the body of the paper assumes that different levels of lobbying effort result in different \textit{decision-makers}. In the unitary model of DGH97, one decision maker makes different decisions depending on the level of lobbying effort. This distinction is mainly important in the context of the repeated game. Whereas the unitary decision-maker evaluates future welfare according to the lobbying effort she expects to experience in the future, the non-unitary decision-maker knows that different decision-makers will be in power and evaluates future welfare with her own preferences.
						\item In the auction-menu set-up of the DGH97 model, there does not seem to be a natural alternative to the unitary government interpretation. The lobby offers a contribution schedule of effort levels and tariffs and the government chooses among these pairs. In the non-unitary model, each different effort level determines a different government, so the government cannot choose between elements of a schedule. To analyze this model, I make the standard assumptions of a unitary government and that the lobby has all the bargaining power.
						\item Denote the government's welfare function as $W_G(\tau,\tau^*,e(\tau)) = W(\tau,\tau^*) + g(e(\tau))$ where $W = CS_X + PS_X + CS_Y + PS_Y +TR$ is social welfare. Then do analysis...
					\end{itemize}
				\item Appendix B.3 with analysis of the strong binding case. 
					\begin{itemize}
						\item If the trade agreement involves strong bindings instead of tariff caps (i.e. weak bindings), the legislature must deliver $\tau^a$ and the lobby's optimal effort during a period in which the trade agreement holds is $e_a = 0$. That the lobby pays less---here, nothing---for the protection it receives under the trade agreement is the only change from the base model. This means that nothing changes in the trade war. Likewise the legislature's repeated-game incentive constraint is unchanged. But the lobby's incentive constraint IS changed...
						\item As in MRC, tariff caps in this model serve to keep the lobby `in the game.' A tariff cap makes the lobby's self-enforcement constraint harder to satisfy and thus requires a higher trade agreement tariff for self-enforcement. From the ex-ante point of view, strong bindings are therefore preferable. Tariff caps could be viewed as a way to commit to setting higher trade agreement tariffs and therefore as a mechanism for ensuring that rents are distributed to protectionists ex-post.
					\end{itemize}
			\end{itemize}
	\end{itemize}


\section{Numeraire}
Do I need to add a numeraire so I can put $e$ in terms of it?


\section{Lit review}
\begin{itemize}
	\item Find MRC cites (maybe some in Ethier) for why PE shocks can't be addressed
		\begin{itemize}
			\item Bagwell and Staiger 1999 AER
		\end{itemize}
	\item Have to make connection to endogenous papers clearer
		\begin{itemize}
			\item and make clear that GH is isomorphic / provides microfoundations for Baldwin objective function
			\item More on Coates $\&$ Ludema (2001)
		\end{itemize}
	\item I think that I need to bring people along more carefully from GH/MRC to what I'm doing
	\item Paola: How do I want to sell the paper?
		\begin{itemize}
			\item In relation to what literature?
				\begin{itemize}
					\item What literature does endogenous? Coates-Ludema 2001, MRC, GH
					\item What literature \textit{doesn't}?
				\end{itemize}
			\end{itemize}
	\item Review Bown, Bagwell $\&$ Staiger
	\item Do I include papers like BS2001 (non-repeated) in lit review?
	\item Read Rosendorff $\&$ Milner (2001) [from Cristiane Carneiro at PEIO]
	\item Kyle: McLaren in his chapter calls the Grossman and Helpman setup with
	 contributions 'felonious'
	\item David DeRemer: Safeguards have dropped off in use around the world
		\begin{itemize}
			\item Look at U.S. vs. ROW (he sent papers)
		\end{itemize}
	\item Limao and Tovar 2011
		\begin{enumerate}
			\item Their 2nd (and final) stage is Nash bargaining. Is there an isomorphism to my setup?
			\item It IS clear to take their first, limited case of commitment to tariff cap only (ignore NTB) as analogy for gov't in my model choosing commitment to different tariff level. Need to answer Q1 above in order to answer points 3 and 4:
			\item Their gains from commitment derive from improvement in bargaining power vis-a-vis the lobby (Schelling conjecture invoked). How does this relate to the gain from commitment in my model?
			\item Top of page 6: ``large enough bargaining power.'' Can I make connection to $\ga(e)$?
			\item Note: in their equation 3, $C$ enters expression for NBS's ($\tau$) in equilibrium, it doesn't disappear as in GH
		\end{enumerate}
\end{itemize}


\section{More on escape clause / WEC}
		\begin{itemize}
			\item The part I really liked was when you analysed the exogenous and endogenous effects on gamma simultaneously around page 28.  In fact I was itching to get to that part right from when you discussed it in the introduction.  
				\begin{itemize}
					\item The paper makes the extremely important point almost in passing.  I would be interested to see a more detailed examination of how this works.  It's particularly interesting that in the presence of a shock the governments will use the escape clause as intended but without a shock a lobby group would always `step into the breach' and cause it to be abused.  I would like to see the paper devote more attention to the conditions under which this problem arises and when it does not.
					\item Then the paper begins to address the equally important question of what can be done about this problem, but again stops short of providing a really compelling answer.  I think it would be worth exploring this in detail.  The paper does get to the \textbf{heart of the issue, which is that `s' can be observed by the DSB but e cannot}.  Get to the bottom of how to design a dispute settlement system that resolves the conflict created by `e vs s'.
				\end{itemize}
			\item `Withdrawal of equivalent concessions' (WEC).
				\begin{itemize}
					\item Zissimos (2007): a government \textbf{`chooses the severity of its own punishment'} by the extent of its initial deviation. I'm wondering whether your framework offers an answer for why WEC made sense as an approach to punishment.
					\item Under WEC, it is worth deviating from the agreement in proportion to $s$ because your partner deviates to the same extent and that keeps the agreement on track; the point first made by Bagwell and Staiger (1990).
					\item But WEC might eliminate the incentive to respond to lobbying pressure $e$ because when your trade partner deviates by WEC this takes away from you exactly (in a symmetrical framework) what you gained from the lobby for implementing a deviation of that size.
						\begin{itemize}
							\item \textbf{Think about how WEC may mitigate LOBBY'S incentives through reaction of gov't}
							\item Ben doesn't have any need for escape; have to put WEC into my framework
							\item Government feels $\ga$ the same whether it's elevated because of $s$ or $e$
							\item What is the neutralizing that happens? Why would the government invoke EC when it knows that WEC is coming anyway? Because it's in another sector where it puts less weight right now?
							\item WEC should work against TOT, not PE shock?
						\end{itemize}
					\item I would find an examination of WEC much more compelling that the approach to punishment that you currently discuss on page 20, whereby two bindings are negotiated.
				\end{itemize}
			\item Puzzle to explain: WEC has an effect here, and that effect is weakened by the change in EC rules during the Uruguay Round, even though no retaliation for 3 yrs
				\begin{itemize}
					\item ``Why Are Safeguards under the WTO So Unpopular?'' (World Trade Review 2002), %\url{C:\Users\Kristy\Dropbox\Research\PEshocks\Literature\2016}
					\item Need to understand better what changed, both with safeguard and with other policy instruments
					\item Note that safeguards agreement states that safeguard can only be at level necessary to remedy injury, and must liberalize as possible
					\item \textbf{perhaps better able to respond to $\ga(e)$ before dispute settlement?}
						\begin{itemize}
							\item Lack of effective enforcement before WTO?
						\end{itemize}
				\end{itemize}
			\item What does WTO \textit{really} want? To discourage rent-seeking lobbying but allow governments to escape when there's a real shock.
				\begin{itemize}
					\item There may be legitimate lobbying to communicate about the shock, so can't look at the presence of lobbying as a sufficient statistic
				\end{itemize}
			\item Need to go to continuous value of EC tariff
				\begin{itemize}
					\item There will be optimal tariff for whatever value of $s$ is realized
						\begin{itemize}
							\item Not sure how fine I want to go with $s$
						\end{itemize}
					\item Lobby will choose its optimal level of $e$. Will this be constrained in any way? Probably not, if there isn't a fixed EC-binding.
				\end{itemize}
			\item After $s$ realized, $e$ choosen by lobby. Then government can choose $\tau$
				\begin{enumerate}
					\item consistent with $s$, no retaliation
					\item consistent with $s$ and $e$ and be retaliated against according to WEC (need to work WEC into model--repeated game incentives necessary)
						\begin{itemize}
							\item Is responding to $e$ in some cases worth it?
							\item Does WEC help or hurt vs. grim trigger, $T$ period Nash reversion?
						\end{itemize}
				\end{enumerate}
		\end{itemize}


\section{Costly state verification}
		\begin{itemize}
				\item Have you seen the work that Beshkar and Bond have done on what they call `contingent protection'?  I can't remember if they are drawing on the work of other people when they refer to the escape clause as `costly state verification.'  This seems to provide an alternative way of separating genuine shocks from those created by lobby groups.  How does your approach relate to costly state verification?
		\end{itemize}


\section{Dynamic use constraint}
When would lobby exert effort to top up?
\begin{itemize}
	\item Do I want correlation between endogenous/exogenous parts? 
\end{itemize}


\section{Motivation}
		\begin{itemize}
			\item What it's missing: intersection with reality
				\begin{itemize}
					\item Anecdotes
					\item How policy is used
						\begin{itemize}
							\item I thought that escape clause implications \textit{were} this
						\end{itemize}
					\item Something like how GH predicts export taxes but it doesn't work that way in the real world
					\item Well articulated question / paradox
				\end{itemize}
			\item Look at ``Theory of Managed Trade'' model motivation (BS)
			\item Intro doesn't reflect body of paper
			\item What kinds of empirical questions might we ask with this model?
			\item What would be different because we use this model?
			\item Starting from the introduction, I think you make a convincing case that it is important to endogenize the extent of political pressure and I believe this would be THE selling point of the paper. However, you have not completely sold your contribution and its novelty that you bring in the issue of the objective function of the government. Is this aspect crucial for the point you want to make? If yes, it should be integrate with the previous object from the beginning. If not, it should be postponed a bit.
			\item I also find the introduction too long. You could/should split the literature review that begins on page 4 to a separate section. And in any case, I found that you don't engage with the literature in terms of how you contribute/differentiate from it. Again, your main point seems to get lost into the discussion instead of standing out.
			\item You need to think a bit more about the objective(s) of the paper and how best to achieve them.
			\item Intro: GH gives micro foundations. But there are real consequences of endogenous politics
		\end{itemize}
		

\section{Big picture of which changes do what}
\vskip.2in
		\begin{tabular}{|l|c|c|}
			 & rigid & escape \\
			BS2005  & on-schedule (truthtelling): trivial; & need cost  \\
			 & off (repeated): optimal static if patient enough & \\
			$\ga(e)$ & on: trivial& no need, but revisit side payments\\
			 & off: mabye not b/c of lobby & \\
			$\ga(e,s)$ & ignore for now & is $s$ verifiable? def not w/$\ga(s,e)$
		\end{tabular}
\vskip.2in
\begin{itemize}
	\item Make sure this story is clear in text: I establish baseline tariff cap case, then add escape clause. We can already see a story of $\ldots$ emerge...
	\item Need to be clear when it is flexibility, enforcement, endogenous $\ga$ that screws things up. THEN need to bmake sure it comes out in paper
	\item May want to ditch repeated game part...
\end{itemize}



\section{Structured differently from most papers}
		\begin{itemize}
			\item How did Devashish structure his AER paper: the argument, the model?
				\begin{itemize}
					\item This paper is structured differently than the norm---is there a good reason to do it?
					\item Don't want to challenge what people are used to unless I really need to
					\item In general, how do AER papers \textit{look}?
				\end{itemize}
			\item If above I suggest to add a section, I think that the paper is already too long and divided into too many sections but that goes back to my first comment about identifying the key issue/contribution: endogeneity of political pressure and/or objective function. The analysis is also split between strict and weak commitments. Which scenario clearly develops the point/contribution you want to make? Shouldn't the other one occupy a smaller place (i.e. space) of the paper?)
		\end{itemize}


\section{One good idea / do too much}
		\begin{itemize}
			\item You can have one good idea in a paper, the rest get lost
			\item At the moment, I find the paper a bit unstructured in that it does a lot but it doesn't sell in a neat and clear way what it is doing.
			\item The paper is already too long and divided into too many sections but that goes back to my first comment about identifying the key issue/contribution: endogeneity of political pressure and/or objective function.
			\item So I'm wondering whether you could dispense with most of the static analysis and get into the repeated game more quickly.
		\end{itemize}


\section{Take out repeats}
	Especially of references to BS2005: In some parts the paper also seem repeating itself: you review Bagwell and Staiger (2005) in Section 2.2 but you provide another summary of the same paper in Section 3. Again, a revised structure may reduce the overlap, make the paper shorter, and be more to the point.
	

\section{Misc.}
\begin{itemize}
	\item Define $\ga$ before using it
	\item If add back in assumption 1, add Ethier footnote
	\item standardize $\ga(e,s)$, $\ga(e)$, $\ga(s)$ throughout the text
	\item Peter: $\ga(e)$ and $\ga(s)$ combination is interesting, so it dynamic choice of protection over time, gov't turnover [add to conclusion, it's in slides]
	\item Ben: I'm wondering how you view the time-frame over which your model is set.  On one hand a specific-factors set-up is normally associated with the short-run.  But on the other hand you have an infinite time horizon which suggests very long run.
	\item Clarify information structure
	\item Move footnote about TOT externality (fn:tot) into text and make paragraphs flow together better
	\item Add ref to JIE R$\&$R to fn:krw ?
	\item Expand Proposition 2 (res:repeated) to compare to exogenous case, say what $\tau^R_{W,e}$ IS. Perhaps not all of that IN the proposition.
	\item Need more sign-posting in the three last paragraphs of Section 4.2. Maybe make Section 4.3 a new top-level section.
\end{itemize}


\section{Make a bigger deal of the idea of \texorpdfstring{$\gamma$}{gamma} decreasing function of tariffs---move out of footnote}
Ethier 2012 for protection decreasing function of $\tau$ [NEED TO REVISIT IN TEXT]. ``The Political-Support Approach to Protection,'' Global Journal of Economics
\begin{itemize}
	\item Making $W$ quadratic in $\ga$ such as $\ga - \ga^2$ gives a nice interior max
	\item I think what I have already has returns to $e$ decreasing in $\tau$. Don't I show it? If not in this paper, in RoIE paper.
	\item What embodies returns to lobbying?
		\begin{itemize}
			\item $e \rightarrow \ga(e)$ is how lobbying is translated into weight in political process. It's not `returns.'
			\item $\tau$ isn't even returns. $\pi(\tau(e_2)) - \pi(\tau(e_1))$ is returns.
		\end{itemize}
	\item To get returns decreasing in $\tau$, maybe weighting can't be confined to $PS$
\end{itemize}


\section{Lobby's optimal effort}
Need to be clear about when lobby's optimal $\tau$ is higher than gov'ts: that is, when TA would want to rein in lobby and escape clause would ruin that


\section{Add empirics / evidence?}


\section{Existence proofs?}

\newpage
Current structure of paper (August 25, 2016)
\begin{enumerate}
	\item Introduction
	\item Model
	\item Rigid Tariffs with Endogenous Political Pressure
		\begin{itemize}
			\item[3.1] Perfect External Enforcement. Proposition 1: weak bindings and ext. enforcement imply applied tariff = binding and may use binding to encourage or restrain lobbying
			\item[3.2] Self-Enforcing Trade Agreements
				\begin{itemize}
					\item[3.2.1] Repeated Game
					\item[3.2.2] Prop 2: No ext enforcement: self enforcing implies $\tau^a \leq$ optimal binding with external enforcement
				\end{itemize}
		\end{itemize}
	\item Endogenous Political Pressure and the Escape Clause
		\begin{itemize}
			\item[4.1] Strong bindings: no cost when $\ga(e)$ only
			\item[4.2] Side payments: spirit not upheld, but IC
			\item[4.3] $\ga(s,e)$, Prop 3: lower binding never used
			\item[4.4] EC for endogenous politics
		\end{itemize}
	\item Conclusion
\end{enumerate}

\end{document}