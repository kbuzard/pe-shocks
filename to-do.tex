\documentclass[12pt]{article}

\addtolength{\textwidth}{1.4in}
\addtolength{\oddsidemargin}{-.7in} %left margin
\addtolength{\evensidemargin}{-.7in}
\setlength{\textheight}{8.5in}
\setlength{\topmargin}{0.0in}
\setlength{\headsep}{0.0in}
\setlength{\headheight}{0.0in}
\setlength{\footskip}{.5in}
\renewcommand{\baselinestretch}{1.0}
\setlength{\parindent}{0pt}
\linespread{1.1}

\usepackage{amssymb, amsmath, amsthm, bm}
\usepackage[normalem]{ulem}
\usepackage{graphicx,csquotes,verbatim}
\usepackage[backend=biber,block=space,style=authoryear]{biblatex}
\setlength{\bibitemsep}{\baselineskip}
\usepackage[american]{babel}
%dell laptop
\addbibresource{C:/Users/Kristy/Dropbox/Research/xBibs/tradeagreements.bib}
%\addbibresource{C:/Users/Kristy/Documents/Dropbox/Research/xBibs/tradeagreements.bib}
\renewcommand{\newunitpunct}{,}
\renewbibmacro{in:}{}


\usepackage[pdftex,
bookmarks=true,
bookmarksnumbered=false,
pdfview=fitH,
bookmarksopen=true,hyperfootnotes=false]{hyperref}

\DeclareMathOperator*{\argmax}{arg\,max}
\usepackage{xcolor}
\hbadness=10000

\newcommand{\ve}{\varepsilon}
\newcommand{\ov}{\overline}
\newcommand{\un}{\underline}
\newcommand{\ta}{\theta}
\newcommand{\al}{\alpha}
\newcommand{\expect}{\mathbb{E}}
\newcommand{\ga}{\gamma}
\newcommand{\Ga}{\Gamma}
\newcommand{\de}{\delta}

\begin{document}
\begin{center}
PE Shocks To-Do list
\end{center}



\section{Existence proofs?}


\section{Dynamic use constraint}
When would lobby exert effort to top up?
\begin{itemize}
	\item Do I want correlation between endogenous/exogenous parts? 
\end{itemize}


\section{Add export lobby}
		\begin{itemize}
			\item Extension or appendix b/c want to make comparison to existing literature.
		\end{itemize}

\section{Lit review}
\begin{itemize}
	\item Have to make connection to endogenous papers clearer
		\begin{itemize}
			\item More on Coates $\&$ Ludema (2001)
			\item Check out Ben and James new paper
		\end{itemize}
	\item Paola: How do I want to sell the paper?
		\begin{itemize}
			\item In relation to what literature?
				\begin{itemize}
					\item What literature does endogenous? Coates-Ludema 2001, MRC, GH
					\item What literature \textit{doesn't}?
				\end{itemize}
			\end{itemize}
	\item Review Bown, Bagwell $\&$ Staiger
	\item Kyle: McLaren in his chapter calls the Grossman and Helpman setup with contributions `felonious'
	\item David DeRemer: Safeguards have dropped off in use around the world
		\begin{itemize}
			\item Look at U.S. vs. ROW (he sent papers)
			\item Wouters and Zissimos (2016 WP): Appellate Body failed to give clear direction in cases in early 2000s as to when Safeguards can be applied. See Sykes (2003).
		\end{itemize}
	\item Read Rosendorff $\&$ Milner (2001) [from Cristiane Carneiro at PEIO]
	\item Find MRC cites (maybe some in Ethier) for why PE shocks can't be addressed
		\begin{itemize}
			\item Bagwell and Staiger 1999 AER
		\end{itemize}
\end{itemize}


\section{Misc}
\begin{itemize}
	\item Ben: I'm wondering how you view the time-frame over which your model is set.  On one hand a specific-factors set-up is normally associated with the short-run.  But on the other hand you have an infinite time horizon which suggests very long run.
	\item Expand Proposition 2 (res:repeated) to compare to exogenous case, say what $\tau^R_{W,e}$ IS. Perhaps not all of that IN the proposition.
\end{itemize}


\section{More on escape clause / WEC}
		\begin{itemize}
			\item The part I really liked was when you analysed the exogenous and endogenous effects on gamma simultaneously around page 28.  In fact I was itching to get to that part right from when you discussed it in the introduction.  
				\begin{itemize}
					\item The paper makes the extremely important point almost in passing.  I would be interested to see a more detailed examination of how this works.  It's particularly interesting that in the presence of a shock the governments will use the escape clause as intended but without a shock a lobby group would always `step into the breach' and cause it to be abused.  I would like to see the paper devote more attention to the conditions under which this problem arises and when it does not.
				\end{itemize}
			\item `Withdrawal of equivalent concessions' (WEC).
				\begin{itemize}
					\item Zissimos (2007): a government \textbf{`chooses the severity of its own punishment'} by the extent of its initial deviation. I'm wondering whether your framework offers an answer for why WEC made sense as an approach to punishment.
					\item Under WEC, it is worth deviating from the agreement in proportion to $s$ because your partner deviates to the same extent and that keeps the agreement on track; the point first made by Bagwell and Staiger (1990).
					\item But WEC might eliminate the incentive to respond to lobbying pressure $e$ because when your trade partner deviates by WEC this takes away from you exactly (in a symmetrical framework) what you gained from the lobby for implementing a deviation of that size.
						\begin{itemize}
							\item \textbf{Think about how WEC may mitigate LOBBY'S incentives through reaction of gov't}
							\item Ben doesn't have any need for escape; have to put WEC into my framework
							\item Government feels $\ga$ the same whether it's elevated because of $s$ or $e$
							\item What is the neutralizing that happens? Why would the government invoke EC when it knows that WEC is coming anyway? Because it's in another sector where it puts less weight right now?
							\item WEC should work against TOT, not PE shock?
						\end{itemize}
					\item I would find an examination of WEC much more compelling that the approach to punishment that you currently discuss on page 20, whereby two bindings are negotiated.
				\end{itemize}
			\item Puzzle to explain: WEC has an effect here, and that effect is weakened by the change in EC rules during the Uruguay Round, even though no retaliation for 3 yrs
				\begin{itemize}
					\item ``Why Are Safeguards under the WTO So Unpopular?'' (World Trade Review 2002), %\url{C:\Users\Kristy\Dropbox\Research\PEshocks\Literature\2016}
					\item Need to understand better what changed, both with safeguard and with other policy instruments
					\item Note that safeguards agreement states that safeguard can only be at level necessary to remedy injury, and must liberalize as possible
					\item \textbf{perhaps better able to respond to $\ga(e)$ before dispute settlement?}
						\begin{itemize}
							\item Lack of effective enforcement before WTO?
						\end{itemize}
				\end{itemize}
			\item What does WTO \textit{really} want? To discourage rent-seeking lobbying but allow governments to escape when there's a real shock.
				\begin{itemize}
					\item There may be legitimate lobbying to communicate about the shock, so can't look at the presence of lobbying as a sufficient statistic
				\end{itemize}
			\item Does WEC help or hurt vs. grim trigger, $T$ period Nash reversion?
		\end{itemize}


\newpage
Current structure of paper (August 25, 2016)
\begin{enumerate}
	\item Introduction
	\item Model
	\item Rigid Tariffs with Endogenous Political Pressure
		\begin{itemize}
			\item[3.1] Perfect External Enforcement. Proposition 1: weak bindings and ext. enforcement imply applied tariff = binding and may use binding to encourage or restrain lobbying
			\item[3.2] Self-Enforcing Trade Agreements
				\begin{itemize}
					\item[3.2.1] Repeated Game
					\item[3.2.2] Prop 2: No ext enforcement: self enforcing implies $\tau^a \leq$ optimal binding with external enforcement
				\end{itemize}
		\end{itemize}
	\item Endogenous Political Pressure and the Escape Clause
		\begin{itemize}
			\item[4.1] Strong bindings: no cost when $\ga(e)$ only
			\item[4.2] Side payments: spirit not upheld, but IC
			\item[4.3] $\ga(s,e)$, Prop 3: lower binding never used
			\item[4.4] EC for endogenous politics
		\end{itemize}
	\item Conclusion
\end{enumerate}



\end{document}