\documentclass[12pt]{article}

\addtolength{\textwidth}{1.4in}
\addtolength{\oddsidemargin}{-.7in} %left margin
\addtolength{\evensidemargin}{-.7in}
\setlength{\textheight}{8.5in}
\setlength{\topmargin}{0.0in}
\setlength{\headsep}{0.0in}
\setlength{\headheight}{0.0in}
\setlength{\footskip}{.5in}
\renewcommand{\baselinestretch}{1.0}
\setlength{\parindent}{0pt}
\linespread{1.1}

\usepackage{amssymb, amsmath, amsthm, bm}
\usepackage[normalem]{ulem}
\usepackage{graphicx,csquotes,verbatim}
\usepackage[backend=biber,block=space,style=authoryear]{biblatex}
\setlength{\bibitemsep}{\baselineskip}
\usepackage[american]{babel}
%dell laptop
\addbibresource{C:/Users/Kristy/Dropbox/Research/xBibs/tradeagreements.bib}
%\addbibresource{C:/Users/Kristy/Documents/Dropbox/Research/xBibs/tradeagreements.bib}
\renewcommand{\newunitpunct}{,}
\renewbibmacro{in:}{}


\usepackage[pdftex,
bookmarks=true,
bookmarksnumbered=false,
pdfview=fitH,
bookmarksopen=true,hyperfootnotes=false]{hyperref}

\DeclareMathOperator*{\argmax}{arg\,max}
\usepackage{xcolor}
\hbadness=10000

\newcommand{\ve}{\varepsilon}
\newcommand{\ov}{\overline}
\newcommand{\un}{\underline}
\newcommand{\ta}{\theta}
\newcommand{\al}{\alpha}
\newcommand{\expect}{\mathbb{E}}
\newcommand{\ga}{\gamma}
\newcommand{\Ga}{\Gamma}
\newcommand{\de}{\delta}

\begin{document}
\begin{center}
PE Shocks To-Do list
\end{center}

March 30, 2016

\section{Motivation}
		\begin{itemize}
			\item What it's missing: intersection with reality
				\begin{itemize}
					\item Anecdotes
					\item How policy is used
						\begin{itemize}
							\item I thought that escape clause implications \textit{were} this
						\end{itemize}
					\item Something like how GH predicts export taxes but it doesn't work that way in the real world
					\item Well articulated question / paradox
				\end{itemize}
			\item Look at ``Theory of Managed Trade'' model motivation (BS)
			\item Intro doesn't reflect body of paper
			\item What kinds of empirical questions might we ask with this model?
			\item What would be different because we use this model?
			\item Starting from the introduction, I think you make a convincing case that it is important to endogenize the extent of political pressure and I believe this would be THE selling point of the paper. However, you have not completely sold your contribution and its novelty that you bring in the issue of the objective function of the government. Is this aspect crucial for the point you want to make? If yes, it should be integrate with the previous object from the beginning. If not, it should be postponed a bit.
			\item I also find the introduction too long. You could/should split the literature review that begins on page 4 to a separate section. And in any case, I found that you don't engage with the literature in terms of how you contribute/differentiate from it. Again, your main point seems to get lost into the discussion instead of standing out.
			\item You need to think a bit more about the objective(s) of the paper and how best to achieve them.
			\item Intro: GH gives micro foundations. But there are real consequences of endogenous politics
		\end{itemize}
		
\newpage
\section{More on escape clause / WEC}
		\begin{itemize}
			\item The part I really liked was when you analysed the exogenous and endogenous effects on gamma simultaneously around page 28.  In fact I was itching to get to that part right from when you discussed it in the introduction.  
				\begin{itemize}
					\item The paper makes the extremely important point almost in passing.  I would be interested to see a more detailed examination of how this works.  It's particularly interesting that in the presence of a shock the governments will use the escape clause as intended but without a shock a lobby group would always `step into the breach' and cause it to be abused.  I would like to see the paper devote more attention to the conditions under which this problem arises and when it does not.
					\item Then the paper begins to address the equally important question of what can be done about this problem, but again stops short of providing a really compelling answer.  I think it would be worth exploring this in detail.  The paper does get to the \textbf{heart of the issue, which is that `s' can be observed by the DSB but e cannot}.  Get to the bottom of how to design a dispute settlement system that resolves the conflict created by `e vs s'.
				\end{itemize}
			\item `Withdrawal of equivalent concessions' (WEC).
				\begin{itemize}
					\item Zissimos (2007): a government \textbf{`chooses the severity of its own punishment'} by the extent of its initial deviation. I'm wondering whether your framework offers an answer for why WEC made sense as an approach to punishment.
					\item Under WEC, it is worth deviating from the agreement in proportion to $s$ because your partner deviates to the same extent and that keeps the agreement on track; the point first made by Bagwell and Staiger (1990).
					\item But WEC might eliminate the incentive to respond to lobbying pressure $e$ because when your trade partner deviates by WEC this takes away from you exactly (in a symmetrical framework) what you gained from the lobby for implementing a deviation of that size.
						\begin{itemize}
							\item \textbf{Think about how WEC may mitigate LOBBY'S incentives through reaction of gov't}
							\item Ben doesn't have any need for escape; have to put WEC into my framework
							\item Government feels $\ga$ the same whether it's elevated because of $s$ or $e$
							\item What is the neutralizing that happens? Why would the government invoke EC when it knows that WEC is coming anyway? Because it's in another sector where it puts less weight right now?
							\item WEC should work against TOT, not PE shock?
						\end{itemize}
					\item I would find an examination of WEC much more compelling that the approach to punishment that you currently discuss on page 20, whereby two bindings are negotiated.
				\end{itemize}
			\item Puzzle to explain: WEC has an effect here, and that effect is weakened by the change in EC rules during the Uruguay Round, even though no retaliation for 3 yrs
				\begin{itemize}
					\item ``Why Are Safeguards under the WTO So Unpopular?'' (World Trade Review 2002), %\url{C:\Users\Kristy\Dropbox\Research\PEshocks\Literature\2016}
					\item Need to understand better what changed, both with safeguard and with other policy instruments
					\item Note that safeguards agreement states that safeguard can only be at level necessary to remedy injury, and must liberalize as possible
					\item \textbf{perhaps better able to respond to $\ga(e)$ before dispute settlement?}
						\begin{itemize}
							\item Lack of effective enforcement before WTO?
						\end{itemize}
				\end{itemize}
			\item What does WTO \textit{really} want? To discourage rent-seeking lobbying but allow governments to escape when there's a real shock.
				\begin{itemize}
					\item There may be legitimate lobbying to communicate about the shock, so can't look at the presence of lobbying as a sufficient statistic
				\end{itemize}
			\item Need to go to continuous value of EC tariff
				\begin{itemize}
					\item There will be optimal tariff for whatever value of $s$ is realized
						\begin{itemize}
							\item Not sure how fine I want to go with $s$
						\end{itemize}
					\item Lobby will choose its optimal level of $e$. Will this be constrained in any way? Probably not, if there isn't a fixed EC-binding.
				\end{itemize}
			\item After $s$ realized, $e$ choosen by lobby. Then government can choose $\tau$
				\begin{enumerate}
					\item consistent with $s$, no retaliation
					\item consistent with $s$ and $e$ and be retaliated against according to WEC (need to work WEC into model--repeated game incentives necessary)
						\begin{itemize}
							\item Is responding to $e$ in some cases worth it?
							\item Does WEC help or hurt vs. grim trigger, $T$ period Nash reversion?
						\end{itemize}
				\end{enumerate}
		\end{itemize}
\section{Costly state verification}
		\begin{itemize}
				\item Have you seen the work that Beshkar and Bond have done on what they call `contingent protection'?  I can't remember if they are drawing on the work of other people when they refer to the escape clause as `costly state verification.'  This seems to provide an alternative way of separating genuine shocks from those created by lobby groups.  How does your approach relate to costly state verification?
		\end{itemize}

\newpage
\section{BS2005 vs. MRC2007}
		\begin{itemize}
			\item Dangerous to build on JLS paper; better to connect to bigger cites or will limit impact, where it can be published
				\begin{itemize}
					\item Perhaps higher level model like BS GE model or some alternative approach that is not so rinky dink (my words)
				\end{itemize}
			\item I like the way you use Bagwell and Staiger (2005) as your benchmark.  But I also wondered whether Maggi and Rodriguez-Clare (2007, henceforth MRC) also offered a useful benchmark.  I know you did have comparisons to their paper in many places but it would be useful if this could be more systematic, and if your paper could draw careful parallels to your assumptions and theirs.  In both papers, the government wants to set a tariff cap in order to be able to extract rents from lobbies afterwards.  But your mechanisms seem quite different.  They have capital mobility whereas your factors are specific to each sector.  If capital is highly immobile they have a hold-up problem, which creates an incentive for lobbies to try to prevent too much liberalization under the agreement.  What drives this feature in your model?  Here I think it would be useful to explain carefully the sequence of events in your set-up more carefully.  What is the relationship between lobby effort level and tariff policy?  Is effort fixed first, after which the government sets its tariff?  If so it seems to me that would drive a kind of hold-up because lobbying effort is sunk.  It would be useful to be clearer about this so we can really understand where the results are coming from and how they compare to MRC.  Having done that, you could bring in ex-ante lobbying and treat it the same way. 
			\item In general it seems to me that you seem to "dependent" on Bagwell and Staiger... (e.g. beginning of Sections 4 and 6).
		\end{itemize}
\section{Structured differently from most papers}
		\begin{itemize}
			\item How did Devashish structure his AER paper: the argument, the model?
				\begin{itemize}
					\item This paper is structured differently than the norm---is there a good reason to do it?
					\item Don't want to challenge what people are used to unless I really need to
					\item In general, how do AER papers \textit{look}?
				\end{itemize}
			\item If above I suggest to add a section, I think that the paper is already too long and divided into too many sections but that goes back to my first comment about identifying the key issue/contribution: endogeneity of political pressure and/or objective function. The analysis is also split between strict and weak commitments. Which scenario clearly develops the point/contribution you want to make? Shouldn't the other one occupy a smaller place (i.e. space) of the paper?)
		\end{itemize}
\section{One good idea / do too much}
		\begin{itemize}
			\item You can have one good idea in a paper, the rest get lost
			\item At the moment, I find the paper a bit unstructured in that it does a lot but it doesn't sell in a neat and clear way what it is doing.
			\item The paper is already too long and divided into too many sections but that goes back to my first comment about identifying the key issue/contribution: endogeneity of political pressure and/or objective function.
			\item So I'm wondering whether you could dispense with most of the static analysis and get into the repeated game more quickly.
		\end{itemize}


\section{Misc.}
\begin{itemize}
	\item Define $\ga$ before using it
	\item If add back in assumption 1, add Ethier footnote
	\item standardize $\ga(e,s)$, $\ga(e)$, $\ga(s)$ throughout the text
\end{itemize}

\section{Deal with ``dynamic preferences'' issue}
		\begin{itemize}
			\item References from Krishna and Sadowski (2014) ECTA
				\begin{itemize}
					\item DILLENBERGER, D., J. S. LLERAS, P. SADOWSKI, AND N. TAKEOKA (2013): ``A Theory of Subjective Learning,'' Working Paper. 
					\item GUL, F., AND W. PESENDORFER (2004): ``Self-Control and the Theory of Consumption,'' Econometrica, 72 (1), 119-158
					\item JONES, R. A., AND J. M. OSTROY (1984): ``Flexibility and Uncertainty,'' Review of Economic Studies, 51 (1), 13-32
					\item KOOPMANS, T. C. (1964): ``On the Flexibility of Future Preferences,'' in Human Judgments and Optimality, ed. by M. W. Shelly and G. L. Bryan. New York: Wiley.
					\item KREPS, D. M., AND E. L. PORTEUS (1978): ``Temporal Resolution of Uncertainty and Dynamic Choice Theory,'' Econometrica, 46 (1), 185-200.
				\end{itemize}
		\end{itemize}
\section{Take out repeats}
	Especially of references to BS2005: In some parts the paper also seem repeating itself: you review Bagwell and Staiger (2005) in Section 2.2 but you provide another summary of the same paper in Section 3. Again, a revised structure may reduce the overlap, make the paper shorter, and be more to the point.
	
\section{Make a bigger deal of the idea of $\ga$ decreasing function of $\tau$---move out of footnote}
\section{Big picture of which changes do what}
\vskip.2in
		\begin{tabular}{|l|c|c|}
			 & rigid & escape \\
			BS2005  & on-schedule (truthtelling): trivial; & need cost  \\
			 & off (repeated): optimal static if patient enough & \\
			$\ga(e)$ & on: trivial& no need, but revisit side payments\\
			 & off: mabye not b/c of lobby & \\
			$\ga(e,s)$ & ignore for now & is $s$ verifiable? def not w/$\ga(s,e)$
		\end{tabular}
\vskip.2in
\begin{itemize}
	\item Make sure this story is clear in text: I establish baseline tariff cap case, then add escape clause. We can already see a story of $\ldots$ emerge...
	\item Need to be clear when it is flexibility, enforcement, endogenous $\ga$ that screws things up. THEN need to bmake sure it comes out in paper
	\item May want to ditch repeated game part...
\end{itemize}

\section{Lit review}
\begin{itemize}
	\item Find MRC cites (maybe some in Ethier) for why PE shocks can't be addressed
		\begin{itemize}
			\item Bagwell and Staiger 1999 AER
		\end{itemize}
	\item Have to make connection to endogenous papers clearer
	\item Review Bown, Bagwell $\&$ Staiger
	\item Do I include papers like BS2001 (non-repeated) in lit review?
	\item Limao and Tovar 2011
		\begin{enumerate}
			\item Their 2nd (and final) stage is Nash bargaining. Is there an isomorphism to my setup?
			\item It IS clear to take their first, limited case of commitment to tariff cap only (ignore NTB) as analogy for gov't in my model choosing commitment to different tariff level. Need to answer Q1 above in order to answer points 3 and 4:
			\item Their gains from commitment derive from improvement in bargaining power vis-a-vis the lobby (Schelling conjecture invoked). How does this relate to the gain from commitment in my model?
			\item Top of page 6: ``large enough bargaining power.'' Can I make connection to $\ga(e)$?
			\item Note: in their equation 3, $C$ enters expression for NBS's ($\tau$) in equilibrium, it doesn't disappear as in GH
		\end{enumerate}
\end{itemize}

\section{Clarify information structure}

\section{Add empirics / evidence?}

\section{Existence proofs?}

\section{Lobby's optimal effort}
Need to be clear about when lobby's optimal $\tau$ is higher than gov'ts: that is, when TA would want to rein in lobby and escape clause would ruin that

\section{Dynamic use constraint}
When would lobby exert effort to top up?
\begin{itemize}
	\item Do I want correlation between endogenous/exogenous parts? 
\end{itemize}

\end{document}