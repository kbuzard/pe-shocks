\documentclass[12pt]{article}

\addtolength{\textwidth}{1.4in}
\addtolength{\oddsidemargin}{-.7in} %left margin
\addtolength{\evensidemargin}{-.7in}
\setlength{\textheight}{8.5in}
\setlength{\topmargin}{0.0in}
\setlength{\headsep}{0.0in}
\setlength{\headheight}{0.0in}
\setlength{\footskip}{.5in}
\renewcommand{\baselinestretch}{1.0}
\setlength{\parindent}{0pt}
\linespread{1.1}

\usepackage{amssymb, amsmath, amsthm, bm}
\usepackage[normalem]{ulem}
\usepackage{graphicx,csquotes,verbatim}
\usepackage[backend=biber,block=space,style=authoryear]{biblatex}
\setlength{\bibitemsep}{\baselineskip}
\usepackage[american]{babel}
%dell laptop
\addbibresource{C:/Users/Kristy/Dropbox/Research/xBibs/tradeagreements.bib}
%\addbibresource{C:/Users/Kristy/Documents/Dropbox/Research/xBibs/tradeagreements.bib}
\renewcommand{\newunitpunct}{,}
\renewbibmacro{in:}{}


\DeclareMathOperator*{\argmax}{arg\,max}
\usepackage{xcolor}
\hbadness=10000

\newcommand{\ve}{\varepsilon}
\newcommand{\ov}{\overline}
\newcommand{\un}{\underline}
\newcommand{\ta}{\theta}
\newcommand{\al}{\alpha}
\newcommand{\expect}{\mathbb{E}}
\newcommand{\ga}{\gamma}
\newcommand{\Ga}{\Gamma}
\newcommand{\de}{\delta}

\begin{document}
\begin{center}
PE Shocks To-Do list
\end{center}

December 30, 2015
\begin{itemize}
	\item From Chad: it's missing intersection with reality
		\begin{itemize}
			\item Well articulated question / paradox
		\end{itemize}
	\item From Ben
		\begin{itemize}
			\item It's particularly interesting that in the presence of a shock the governments will use the escape clause as intended but without a shock a lobby group would always `step into the breach' and cause it to be abused.  I would like to see the paper devote more attention to the conditions under which this problem arises and when it does not.
			\item `Withdrawal of equivalent concessions' (WEC).
				\begin{itemize}
					\item Zissimos (2007): a government `chooses the severity of its own punishment' by the extent of its initial deviation.  In that paper, I took WEC for granted and then played out trade liberalization according to its rules.  But this left open the question of why WEC made sense as an approach to punishment.  Now I'm wondering whether your framework offers an answer.
					\item Under WEC, it is worth deviating from the agreement in proportion to $s$ because your partner deviates to the same extent and that keeps the agreement on track; the point first made by Bagwell and Staiger (1990).
					\item But WEC might eliminate the incentive to respond to lobbying pressure $e$ because when your trade partner deviates by WEC this takes away from you exactly (in a symmetrical framework) what you gained from the lobby for implementing a deviation of that size.
					\item I would find an examination of WEC much more compelling that the approach to punishment that you currently discuss on page 20, whereby two bindings are negotiated: one that applies if the trade partner keeps to the agreement and one that applies if they deviate.  As far as I know this latter approach does not mirror at all the institutional features actually set out by the GATT/WTO.
				\end{itemize}
		\end{itemize}
\end{itemize}

\end{document}