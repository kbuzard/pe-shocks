%\documentclass{beamer} 
\documentclass[handout]{beamer} 
\usetheme{Ilmenau}
\usepackage{graphicx,verbatim,hyperref}
\usepackage{textpos}

\usecolortheme{beaver}
\useinnertheme{default}
\setbeamertemplate{itemize item}[triangle]
\setbeamertemplate{itemize subitem}[triangle]
\setbeamertemplate{itemize subsubitem}[circle]
\setbeamertemplate{enumerate items}[default]
\setbeamertemplate{blocks}[upper=block head,rounded]
\setbeamercolor{item}{fg=black}
\usefonttheme{serif} %should allow ccfonts to take effect

\usepackage{cite}
\usepackage{times, verbatim,xcolor,bm}
%\usepackage[usenames,dvipsnames]{color}
\usepackage{amsbsy,amssymb, amsmath, amsthm}
\usepackage{booktabs}
%David miller's fonts
	\usepackage[T1]{fontenc}
	\usepackage[boldsans]{ccfonts}
	%\usepackage[boldsans]{concmath}
	\usepackage[euler-hat-accent]{eulervm}

\newcommand{\al}{\alpha}
\newcommand{\expect}{\mathbb{E}}
\newcommand{\Bt}{B(\bm{\tau^a})}
\newcommand{\bta}{\bm{\tau^a}}
\newcommand{\btn}{\bm{\tau^{tw}}}
\newcommand{\ga}{\gamma}
\newcommand{\ve}{\varepsilon}
\newcommand{\ta}{\theta}
\newcommand{\de}{\delta}
\newcommand{\ov}{\overline}
\newcommand{\un}{\underline}

\newenvironment{changemargin}[2]{% 
  \begin{list}{}{% 
    \setlength{\topsep}{0pt}% 
    \setlength{\leftmargin}{#1}% 
    \setlength{\rightmargin}{#2}% 
    \setlength{\listparindent}{\parindent}% 
    \setlength{\itemindent}{\parindent}% 
    \setlength{\parsep}{\parskip}% 
  }% 
  \item[]}{\end{list}} 
	
	\let\Tiny=\tiny

\usepackage{appendixnumberbeamer}


\title[Endogenous Politics and the Design of Trade Institutions\hspace{2.05in}\insertframenumber/\inserttotalframenumber]{Endogenous Politics and \\ the Design of Trade Institutions}
%\author[Kristy Buzard]{Kristy Buzard \\ Syracuse University and The Wallis Institute \\ kbuzard@syr.edu}
\author[Kristy Buzard]{\texorpdfstring{Kristy Buzard\newline Syracuse University \newline\url{kbuzard@syr.edu}}{Kristy Buzard}}
\date{January 6, 2017}
\begin{document}
\maketitle
%\insertpresentationendpage removed b/c of appendix




\section{Overview}
\subsection{Preview}
\begin{frame}{The Questions}

\pause
\begin{enumerate}[<+->]
\item When is endogenizing political pressure important for answering optimal design questions?
	\begin{itemize}
		\item Exogenous vs. endogenous politics
	\end{itemize}
	\vskip.1in
\item When do governments want to use trade agreements to manipulate domestic lobbying incentives?
	\begin{itemize}
		\item Government objective function
	\end{itemize}
\end{enumerate}

\end{frame}


\begin{frame}{Preview}
\frametitle{Political Economy of Trade Institutions}
\pause
With a few exceptions, TA design literature has taken political economy forces to be exogenous. I:
\pause
\begin{itemize}[<+->]
	\item endogenize politics into a standard model for studying TA design questions
	\item carefully distinguish between dynamics induced by exogenous and endogenous politics for tariff caps with escape clause
		%\begin{itemize}[<+->]
			%\item base case with tariff caps
			%\item 
		%\end{itemize}
	\item examine escape clause design when both exogenous and endogenous forces are present
\end{itemize}
\end{frame}


\begin{frame}{Preview}
\frametitle{Results}

\pause
\begin{itemize}[<+->]
	\item Show that TAs may be used to manipulate domestic political actors (no long-run distortions)
	\item Escape clause outcomes are very different with endogenous politics
	\item Demonstrate that (standard, theoretical) escape clause can't work in the presence of endogenous political pressure
		\begin{itemize}
			\item Points to real-world design of WTO Agreement on Safeguards
			\item May explain why escape clause has fallen out of use
		\end{itemize}
\end{itemize}
\end{frame}

\begin{comment}
\begin{frame}{Outline of Talk}
\pause
\begin{enumerate}[<+->]
	\item Model
	\item Government Objective Function
	\item Base Model with Tariff Caps
	\item Escape Clause
	\item Conclusion
\end{enumerate}
\end{frame}
\end{comment}








\section{Model}
\subsection{Economic and Political Structure}
\begin{frame}{Economy}
Two countries: home and foreign (${}^*$)
%\pause
\begin{itemize}
	\item Separable in two goods: $X$ and $Y$
			\begin{itemize}
				\item $P_i$: home price of good $i$
				\item $P_i^*$: foreign price of good $i$
			\end{itemize}
	\item Demand identical for both goods in both countries
		\begin{itemize}
			\item $D(P_i) = 1 - P_i$
		\end{itemize}
	\item Supply: $Q_X^*(P_X) > Q_X(P_X)$ $\forall P_X$; symmetric for $Y$ 
		\begin{itemize}
			\item $Q_X(P_X) = \frac{P_X}{2}$; $Q_Y(P_Y) = P_Y$
			\item Home net importer of $X$, net exporter of $Y$
		\end{itemize}
\end{itemize}

\end{frame}

\begin{frame}{Policy and Politics}

\pause
Home levies $\tau$ on $X$, Foreign levies $\tau^*$ on $Y$
\pause
\begin{itemize}
	\item $P_X=P_X^W + \tau$ increasing in $\tau$
	\pause
	\item $\pi_X(P_X)$ increasing in $P_X$, therefore also $\tau$
\end{itemize}

\pause
\vskip.2in
Non-tradable specific factors motivate political activity
\end{frame}



\begin{frame}{Timeline}
\pause
%\vskip.1in
Each period:
\pause
\begin{enumerate}[<+->]
	\item {\bfseries Trade Agreement Formed}
		\begin{enumerate}[i.]
			%\pause
			\item Governments set trade policy in international agreement
		\end{enumerate}
	%\pause
	\item \textbf{Domestic Politics Played Out}
		\begin{enumerate}[i.]
			%\pause
			\item Exogenous shocks are realized AND/OR
			%\pause
			\item Import-competing industry lobbies government for protection 
		\end{enumerate}
	%\pause
	\item \textbf{Tariffs are Applied}
	%\pause
		\begin{enumerate}[i.]
			\item Given political pressure, governments choose applied tariff levels
		\end{enumerate}
\end{enumerate}
\end{frame}


\subsection{The Players}

\begin{frame}{Applied Tariff Decision}
\pause
  Baldwin-style government objective function:
\pause
\[
  W = \mathit{CS}_X(\tau) + \ga(s,e) \pi_X(\tau) + \mathit{CS}_Y(\tau^*) + \pi_Y(\tau^*) + \mathit{TR}(\tau) - e
\]
\vskip-.1in
\pause
\begin{itemize}[<+->]
	\item Standard \textit{except} for $s$ and $e$: 
		\begin{itemize}
			\item $s$: exogenous shock
			\item $e$: lobbying effort
		\end{itemize}
	\item Optimal applied tariff is a function of $\ga(s,e)$
		\begin{itemize}
			\item Ignores foreign welfare
			\item Takes into account trade agreement enforcement
		\end{itemize}
	%\item Assume $\ga$, $\ga^*$ is private info of each government
\end{itemize}
\end{frame}

\begin{frame}
\frametitle{Domestic Political Pressure}
Two potential sources
\pause
\begin{enumerate}[<+->]
	\item Exogenous shocks
		\begin{itemize}[<+->]
			\item Shock directly to $\ga$ as in Bagwell $\&$ Staiger (2005): $\ga$, $\ga^*$ with CDF $H(\ga)$ on support $\left[\un{\ga},\ov{\ga}\right]$; or
			\item Can take $\ga$ as a function of $s$: $\ga(s)$
		\end{itemize}
	
	\item Endogenous effort choice of lobby, $e$
		\begin{itemize}[<+->]
			\item Lobby chooses effort to maximize profits, $\pi(\cdot)$, net of lobbying effort, $e$
			\item Call lobby's optimal effort choice $e^L$
						\[
						  e^L = \max_e \pi(\tau(\ga(e))) - e
						\]
		\end{itemize}
\end{enumerate}

\end{frame}


\begin{frame}{Trade Agreement Negotiation}
	\pause
Model as Nash bargain between the two countries' governments

	\pause
\begin{itemize}[<+->]
	\item Maximize joint political welfare
	\item Disagreement point: non-cooperative outcome
\end{itemize}


\vskip.2in
\pause
Once agreement is set, cooperation enforced by repeated-game punishments conditioned on history, history + DSB signal

\end{frame}


\section{Objective Fcn}

\begin{comment}
\subsection{Role and Design of TAs}
\begin{frame}{Design of Trade Agreements}
\pause
\begin{itemize}[<+->]
	\item \textbf{Tariff caps}: Bagwell and Staiger 2005, Horn et al 2010, Amador and Bagwell 2012; Beshkar and Bond 2012
	\item \textbf{Escape clause}: Bagwell and Staiger 2005, Horn et al 2010, 
	\item \textbf{Shallow vs. deep integration}: Bagwell and Staiger 2001, DeRemer 2014
	\item \textbf{Dispute settlement}: Maggi 1999, Ludema 2001, Maggi and Staiger 2011/2013, Klimenko et al 2008
	\item \textbf{Property vs. liability rules}: Pauwelyn 2008, Beshkar 2010, Maggi and Staiger 2014
	\item \textbf{Retaliation}: Bown 2002/2004, Beshkar 2010
\end{itemize}
\end{frame} 


\begin{frame}{Role of Trade Agreements: TOT Externality}
\pause
Bagwell and Staiger (2002)
\pause
\begin{itemize}[<+->]
	\item Joint social welfare maximized at free trade
	\item Trade war (i.e. no agreement)
		\begin{itemize}
			\item Maximize with respect to home country welfare only
			\item Terms of trade (TOT) externality $\Rightarrow$ positive tariffs  
		\end{itemize}
	\item Trade agreements
    \begin{itemize}
			\item Now take into account impact on foreign welfare
			\item Internalize TOT externality $\Rightarrow$ free trade
    \end{itemize}
\end{itemize}
\end{frame} 

 
\begin{frame}{Role of Trade Agreements: TOT Externality}
\pause
Grossman and Helpman (1995)
\pause
\begin{itemize}[<+->]
	\item Add endogenous politics
	\item Now in ``Trade War'': two reasons for positive tariff
		\begin{itemize}
			\item TOT externality + pressure from import competing lobby
		\end{itemize}
	
	\item Trade agreement: only internalizes TOT externality
\end{itemize}
\end{frame} 

\begin{comment} 
\begin{frame}{Role of Trade Agreements: New Trade Theory}
\pause
``New Trade Theory'' externalities
\pause
\begin{itemize}[<+->]
	\item Source: imperfect competition (firm level)
		\begin{itemize}
			\item Delocation: Ossa 2011, Bagwell and Staiger 2012
			\item Profit Shifting: Mrazova 2011, Ossa 2012  
		\end{itemize}
	\item Trade agreements
    \begin{itemize}
			\item Now take into account impact on foreign welfare
			\item Internalize these new externalities
    \end{itemize}
\end{itemize}
\end{frame} 


\begin{frame}{Role of Trade Agreements: Domestic Commitment}
\pause
\begin{itemize}[<+->]
	\item Maggi and Rodriguez-Clare (1998, 2007)
		\begin{itemize}
			\item Allow for (imperfect) capital mobility
			\item Domestic investment decisions depend on level of protection
			\item Inability to commit $\Rightarrow$ investment too high b/c importers know protection will respond
			\item Trade agreements provide commitment device
    \end{itemize}
	\item Mitra (2002)
		\begin{itemize}
			\item Here distortion is wasted resources in lobby formation
		\end{itemize}
\end{itemize}
\end{frame} 
\end{comment}

\subsection{Objective Function}
\begin{frame}{Restraining Political Pressure through TAs}
\pause
%It's easy to see that weak bindings keep lobbying alive relative to strong bindings
\begin{itemize}[<+->]
	\item Will TA be used to discourage lobbying? Depends on how gov't welfare varies in $\ga$
	\item With standard Baldwin-style objective function, welfare always increases with $\ga$\spaceskip1pt
\[
  W = \mathit{CS}_X(\tau) + \ga \pi_X(\tau) + \mathit{CS}_Y(\tau^*) + \pi_Y(\tau^*) + \mathit{TR}(\tau)
\] 

	\begin{itemize}
		\item Isomorphic to `Protection for Sale' objective function
	\end{itemize}

	\item When subtracting lobbying effort, welfare no longer monotonic in $\ga$
\end{itemize}
\end{frame}

%\begin{frame}
%\includegraphics[height=2.75in, width=4.25in]{weight.jpg}
%\end{frame}

\begin{comment}
\section{Tariff Caps}
\subsection{Tariff Caps}

\begin{frame}{Tariff Caps: Exogenous vs. Endogenous $\ga$}
\pause
Strong binding: an exact tariff commitment
\begin{itemize}
	\pause
	\item Must apply the precise tariff level specified in TA
\end{itemize}

\pause
\vskip.2in
Weak binding: a tariff cap
\begin{itemize}
\pause
	\item Must set tariff at or below specified level 
\end{itemize}

\pause
\vskip.4in
Perfect external enforcement vs. self-enforcement (repeated-game)

\end{frame}



\begin{frame}{Strong Bindings}
Must set tariff at specified level
\pause
\begin{itemize}[<+->]
	\item $\ga$ exogenous (Bagwell $\&$ Staiger 2005): optimal (rigid) strong binding is the tariff that is politically optimal for the expected realization of $\ga$
  \item $\ga$ endogenous: if governments have no inherent bias toward protection, the lobbies exert no effort and are afforded no protection 
\end{itemize}
\end{frame}
\end{comment}

\begin{comment}
{\begin{frame}{Tariff Caps: Exogenous vs. Endogenous $\ga$}
\pause
Must set tariff at or below specified level (aka tariff cap)
\pause
\begin{itemize}[<+->]
	\item $\ga$ exogenous (Bagwell $\&$ Staiger 2005): Negotiated weak bindings (a) are higher than those gov'ts would choose if they instead negotiated strong bindings and (b) imply that governments with low realizations of $\ga$ set their applied tariffs strictly below the bound level.

	\item $\ga$ endogenous: Governments will not set applied tariffs strictly below the bound level. They may use the weak tariff binding either to encourage and/or restrain endogenous political pressure.

\end{itemize}
\end{frame}


\begin{frame}[label=self]
\frametitle{Tariff Caps with Self Enforcement}
\pause
\begin{itemize}[<+->]
	\item $\ga$ exogenous (Bagwell $\&$ Staiger 2005): if governments patient enough ($\de$ high enough), optimal externally-enforced weak binding can be self-enforced
  \item $\ga$ endogenous: optimal externally-enforced weak binding may not be self-enforcing
		\begin{itemize}
			\item Problem: lobby is an additional repeated-game player
			\item Lobby's incentive constraint is harder to satisfy as $\de$ increases
		\end{itemize}
	 
\end{itemize}
\pause
\hyperlink{repeated<1>}{\beamergotobutton{Repeated Game Intuition}}
\end{frame}
\end{comment}



\begin{comment}
\begin{frame}{Highlight: One-Shot Intuition}
\pause
\textbf{Legislature}
\pause
\begin{itemize}
	\item Breaks agreement if median legislator prefers $\btn$ to $\bta$
\end{itemize}

\pause
\vskip.1in
\textbf{Lobby}
\pause
\begin{itemize}[<+->]
	\item Given the $\bta$ it faces, lobby knows what $e_b$ is required to break the agreement
 	\item Lobby pays if $\bm{\pi}(\btn)$ - lobbying costs $> \pi(\bta)$
\end{itemize}

\pause
\vskip.13in
\textbf{Executives}
\pause
\begin{itemize}[<+->]
	\item Set $\bta$ to make paying $e_b$ unprofitable
		\pause
		\begin{itemize}
			\item[$\Rightarrow$] $e_b=0$, agreement remains in force
		\end{itemize}
	\item High tariffs, no lobbying, no trade disruptions
\end{itemize}
\end{frame}
\end{comment}


\section{Escape Clause}
\subsection{Escape Clause}
\begin{frame}{Escape Clause with Exogenous Politics}

\pause
When $\ga$ is \textit{only} exogenous (Bagwell $\&$ Staiger 2005):

\pause
\begin{itemize}[<+->]
	\item Simple escape clause: add a second (higher) negotiated weak binding
		\begin{itemize}
			\item Escape clause is designed to allow higher applied tariff when realization of $\ga$ is high
		\end{itemize}
	\item Improves political efficiency
	\item Can improve self-enforcement
	%\item Incentive compatibility becomes an issue
\end{itemize}
\end{frame}


\begin{comment}
\begin{frame}{Incentive compatibility}
\pause
Escape clause is meant to allow higher applied tariff when realized $\ga$ is high
\pause
\begin{itemize}[<+->]
	\item $\ga$ is private information
	\item We want truthful revelation, but truth-telling must be in the best interest of each gov't
	\item Gov't can exploit TOT externality by reporting high $\ga$ even when $\ga$ is low
		\begin{itemize}
			\item Only way to prevent this is with some cost of using escape clause
		\end{itemize}
	\end{itemize}
\end{frame}
\end{comment}

\begin{frame}{Escape Clause with Endogenous Politics}
When $\ga$ is \textit{only} endogenous:
\pause
\begin{itemize}[<+->]
	\item Benefit of escape clause from exogenous case is gone
	\item Assuming lower binding is set to maximize political welfare, escape clause encourages inefficiently high lobbying effort / protection
	%\item (Incentive compatibility still an issue, but often not the central one)
		%\begin{itemize}[<+->]
			%\item (If lobby's preferred tariff $\geq$ escape clause binding, gov't experiences high $\ga$, no need to lie)
		%\end{itemize}
  \end{itemize}
	
\pause
\vskip.2in

If $\ga$ is only endogenous, escape clause causes problems, provides no benefits
\end{frame}


\begin{frame}{When the world is more complicated...}
Now suppose political pressure is a result of both endogenous and exogenous forces (i.e. $\ga(s,e)$):
\pause
\begin{itemize}[<+->]
	\item Want escape clause to deal with exogenous shock
	\item But endogenous part $\Rightarrow$ lobbying incentives make it hard to implement escape clause
\end{itemize}

\pause
\vskip.2in
\begin{beamerboxesrounded}[upper=palette tertiary, shadow=true]{Ineffectiveness of Political Criterion for Escape Clause}
  Assume $\ga(s,e) = \ga(s) + \ga(e)$. If an escape clause conditions on $\ga(s,e)$ and $\ga(s^L) < \ga(s^H) < \ga(e^L)$, the lower ``normal'' tariff binding will never be applied.
\end{beamerboxesrounded}
\end{frame}

\begin{frame}{When the world is more complicated... (con't)}
\begin{itemize}[<+->]
	\item To make escape clause work, can't use $\ga$
	\begin{itemize}[<+->]
		\item Need signal of shock that is not influenced by endogenous pressure
	\end{itemize}
	\item Can condition directly on $s$
		\begin{itemize}
			\item This seems to be what the WTO actually \textit{does}
		\end{itemize}
\end{itemize}

\end{frame}


\begin{frame}{An Escape Clause for Endogenous Politics}
\pause
Assume a WTO-like set up: gov't can choose between $\tau^a$, `escape' tariff $\tau(s)$, or politically-optimal $\tau$ matched to $\ga(s,e)$
\pause
\begin{itemize}[<+->]
	\item Assume $s$ verifiable, so no punishment for $\tau(s)$
	\item Punishment for $\tau(\ga(s,e)) > \tau(s)$
\end{itemize}

\vskip.2in
\pause
Optimal $\tau^a$ may lead government to apply $\tau(\ga(s,e))$
\pause
\begin{itemize}[<+->]
	\item When this happens, it leads to dispute, not valid escape
	\item Otherwise, no extra rent-seeking is encouraged
\end{itemize}

\vskip.2in
\pause
May explain why escape clause has fallen out of use

\end{frame}



%\begin{frame}{Optimal Punishments}
%\begin{beamerboxesrounded}[upper=palette tertiary, shadow=true]{Optimal DSI}
%  In the case of political certainty, if non-trivial cooperation is possible in the presence of a lobby, the optimal punishment scheme is finite when the influence of the lobby on legislative preferences is sufficiently strong $\left(\frac{\partial \gamma}{\partial e} \text{ is sufficiently high} \right)$.
%\end{beamerboxesrounded}
%\end{frame}



\section{Conclusion}

\begin{comment}
\begin{frame}{Ex-Ante Lobbying}
Without ex-ante: gov't anticipates ex-post lobbying, chooses $\tau^{\text{TA}}$ that maximizes objective function given $e^{\text{EP}}$
  \[
	  \max_\tau W(\ga(e^{\text{EP}}),\tau) + W^*(\tau)
	\]
\pause
When ex-ante lobbying ($e^{\text{EA}}$) is possible, we could have
\pause
  \[
	  \max_\tau W(\ga(e^{\text{EA}}),\tau) + W^*(\tau)
	\]
	but the lobby would now have to pay \textit{twice}.
\end{frame}
\end{comment}

\begin{frame}{Conclusion}
Taking into account endogenous political forces alongside exogenous ones...
\pause
\begin{itemize}[<+->]
		\item helps explain the structure and enforcement of the WTO Safeguards measure
		\item can help us think about optimal design of trading insitutions
		%\item demonstrates that TAs can be used to discourage lobbing activity in general
		%\item provides additional general explanation for tariff caps
\end{itemize}

\end{frame}

\begin{frame}{Future Work}
\pause
\begin{itemize}[<+->]
	\item Application of framework to other design questions
	\item Interactions between $\ga(s)$ and $\ga(e)$
	\item Choice between protective measures over time
\end{itemize}

\end{frame}

\begin{comment}
\appendix
\begin{frame}[label=repeated]
\frametitle{Repeated Game Intuition}
\pause
Legislature: break agreement if punishment not strong enough
\pause
\begin{itemize}
	\item i.e. if one period of gain from cheater's payoff is greater than T-periods of loss from trade-war
\end{itemize}

\vskip.1in
\pause
Lobby: solve for lowest effort ($\ov{e}_b$) that breaks this constraint
\pause
\begin{itemize}
	\item pay $\ov{e}_b$ if it's less than gain from $T$ periods of trade-war profits
\end{itemize}

\pause
\vskip.1in
Executives: set lowest $\bta$ that makes paying $\ov{e}_b$ unprofitable \textit{and} satisfies legislature's condition
\pause
\begin{itemize}
	\item[$\Rightarrow$] $e_b=0$, agreement remains in force
	\pause
	\item High tariffs, no lobbying, no trade disruptions
\end{itemize}
\hyperlink{self<1>}{\beamergotobutton{Go Back}}
\end{frame}
\end{comment}



\end{document}